\documentclass[11pt]{article}
%-----------Packeges---------------%
\usepackage{amsmath}
\usepackage{amssymb}
\usepackage{amsfonts}
\usepackage{tocloft}
\usepackage{float}
\usepackage{graphicx}
\usepackage[bookmarks=true]{hyperref}
\usepackage{fancyhdr}


%----------Definition & Theorem----%
\newtheorem{definition}{Definition}[subsection]
\newtheorem{theorem}{Theorem}[subsection]
\newtheorem{proposition}{Proposition}[subsection]
\newtheorem{lemma}{Lemma}[subsection]
\newtheorem{corollary}{Corollary}[subsection]

\pagestyle{fancy}
\fancyhead[L]{CS 412}
\fancyhead[C]{HW3}
\fancyhead[R]{Lanxiao Bai}

\usepackage{listings}
\usepackage{color}
\usepackage{enumerate}

\definecolor{dkgreen}{rgb}{0,0.6,0}
\definecolor{gray}{rgb}{0.5,0.5,0.5}
\definecolor{mauve}{rgb}{0.58,0,0.82}

\lstset{frame=tb,
  language=Python,
  aboveskip=3mm,
  belowskip=3mm,
  showstringspaces=false,
  columns=flexible,
  basicstyle={\small\ttfamily},
  numbers=none,
  numberstyle=\tiny\color{gray},
  keywordstyle=\color{black},
  commentstyle=\color{dkgreen},
  stringstyle=\color{black},
  breaklines=true,
  breakatwhitespace=true,
  tabsize=3
}

\begin{document}
	\begin{enumerate}
		\item Considering three sequences
		
		\begin{tabular}{c | l}
			S1 & [A, B, B, B, B, B, C, D, E, A, F, B, C]\\
			S2 & [F, D, D, D, E, E, A, F, D, A, B, Z, C]\\
			S2 & [A, D, F, A, D, E, C, B, F, F, B, A, A, E, A, A, C]
		\end{tabular}
		\begin{itemize}
			\item By definition, we have the result as following table shows
			\begin{center}
			\begin{tabular}{|c | c|}
				\hline
					& Length of minimum occurrence window\\
				\hline
				S1 & 4\\
				\hline
				S2 & 4\\
				\hline
				S3 & 4\\
				\hline
			\end{tabular}
			\end{center}
			\item
			By definition, we have the result as following table shows
			\begin{center}
			\begin{tabular}{|c | c|}
				\hline
					& Number of outliers in outlier based minimum occurrence window\\
				\hline
				S1 & 1\\
				\hline
				S2 & 0\\
				\hline
				S3 & 1\\
				\hline
			\end{tabular}
			\end{center}
		\end{itemize}
		\item According to the instruction of problem 2, the algorithm is implemented in \textit{Question2.lbai5.py}, and we got the following output
		\begin{lstlisting}
1
1, 2
1, 2, 4
1, 4
2
2, 4
2, 4, 5
2, 5
4
4, 5
5
5, 6
6
		\end{lstlisting}
		
	which are all stored in file \textit{output.txt}.
	\end{enumerate}
\end{document}