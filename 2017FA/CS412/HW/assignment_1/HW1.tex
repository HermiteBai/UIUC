\documentclass[11pt]{article}
%-----------Packeges---------------%
\usepackage{amsmath}
\usepackage{amssymb}
\usepackage{amsfonts}
\usepackage{tocloft}
\usepackage{float}
\usepackage{graphicx}
\usepackage[bookmarks=true]{hyperref}
\usepackage{fancyhdr}


%----------Definition & Theorem----%
\newtheorem{definition}{Definition}[subsection]
\newtheorem{theorem}{Theorem}[subsection]
\newtheorem{proposition}{Proposition}[subsection]
\newtheorem{lemma}{Lemma}[subsection]
\newtheorem{corollary}{Corollary}[subsection]

\pagestyle{fancy}
\fancyhead[L]{CS 412}
\fancyhead[C]{HW1}
\fancyhead[R]{Lanxiao Bai}

\usepackage{listings}
\usepackage{color}
\usepackage{enumerate}

\definecolor{dkgreen}{rgb}{0,0.6,0}
\definecolor{gray}{rgb}{0.5,0.5,0.5}
\definecolor{mauve}{rgb}{0.58,0,0.82}

\lstset{frame=tb,
  language=Python,
  aboveskip=3mm,
  belowskip=3mm,
  showstringspaces=false,
  columns=flexible,
  basicstyle={\small\ttfamily},
  numbers=none,
  numberstyle=\tiny\color{gray},
  keywordstyle=\color{black},
  commentstyle=\color{dkgreen},
  stringstyle=\color{black},
  breaklines=true,
  breakatwhitespace=true,
  tabsize=3
}

\begin{document}
	\begin{enumerate}
		\item By running the python code included in file \textit{Question1.lbai5.py}, we get the output as following:
			
	\begin{lstlisting}
count    1000.00000
mean       76.71500
std        13.16355
min        37.00000
25%        68.00000
50%        77.00000
75%        87.00000
max       100.00000
Name: Midterm, dtype: float64
-------------------------------------------
Mode:[77 83]
-------------------------------------------
Variance:173.279054054
	\end{lstlisting}
	
	This result indicate that
		\begin{enumerate}[a.]
			\item Max = 100, Min = 37
			\item Q1 = 68, median = 77, Q3 = 87
			\item mean = 76.715
			\item mode1 = 77 and mode2 = 83
			\item empirical variance = 173.279
		\end{enumerate}
		
		\item By running the python code included in file \textit{Question2.lbai5.py}, we get the output as following:
		
		\begin{lstlisting}
Original empirical variance: 173.279054054
Normalized empirical variance: 1.001001001
-------------------------------------------
Normalized 90:1.00922621869
-------------------------------------------
Pearson's Correlation Coefficient:
          Midterm     Final
Midterm  1.000000  0.544425
Final    0.544425  1.000000
-------------------------------------------
Covariance:
            Midterm       Final
Midterm  173.279054   78.254194
Final     78.254194  119.232176
		\end{lstlisting}
		\begin{enumerate}[a.]
			\item Original empirical variance $>$ Normalized empirical variance
			\item Normalized 90 = 1.009
			\item Pearson's Correlation Coefficient = 0.544
			\item Covariance = 78.254
		\end{enumerate}
		\item 
		\begin{enumerate}[a.]
			\item \[\text{Jaccard Coefficient} = \frac{q}{q + r + s} = \frac{58}{58 + 120 + 2} = 0.322
 \]
			\item By running the python code included in file \textit{Question3.lbai5.py}, we get the output as following:
			\begin{lstlisting}
b.
Norm-1:6152.0
Norm-2:715.327896842
Norm-infty:170.0
---------------------------
c.
Cosine similarity:[[ 0.84140403]]
---------------------------
d.
K-L Divergence:0.207080937332
			\end{lstlisting}
	This result indicate that
				\begin{enumerate}[1.]
					\item When $h = 1$, $\text{dist} = 6152.0$
					\item When $h = 2$, $\text{dist} = 715.328$
					\item When $h = \infty$, $\text{dist} = 170.0$
				\end{enumerate}
				\item Cosine similarity $ = 0.841$
				\item K-L Divergence $ = 0.207$
		\end{enumerate}
		
		\item By running the python code included in file \textit{Question4.lbai5.py}, we get the output as following:
\begin{lstlisting}
chisquare: 2468.183
\end{lstlisting}
	This result indicate that $\chi^2 = 2468.183$
	\end{enumerate}
\end{document}
