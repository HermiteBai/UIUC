\documentclass[11pt]{article}

\usepackage{homeworkpkg}
%% Local Macros and packages: add any of your own definitions here.

\begin{document}

% Homework number, your name and NetID, and optionally, the names/NetIDs of anyone you collaborated with. If you did not collaborate with anybody, leave the last parameter empty.
\homework
    {2}
    {Lanxiao Bai (lbai5)}
    {}

\section*{Problem 1}
\textbf{Solution:} 
	\begin{itemize}
		\item In order to reach the minimum encoding length, we use the strategy that encode the most frequent letter with the least bit and vice versa. As a result, our encoding are as following
		\begin{center}
			\begin{tabular}{c c}
				$A$ & 100\\
				\hline
				$B$ & 101\\
				\hline
				$C$ & 000\\
				\hline
				$D$ & 001\\
				\hline
				$E$ & 010\\
				\hline
				$F$ & 011
			\end{tabular}
		\end{center}
		\item Considering there are $6$ symbols in $T$, if using $2$ bits for each symbol, it is impossible to encode each symbol without making some a prefix of the other. As a result, we need $3$ bits to encode symbols in $T$.
		\item We can calculate entropy
			\[H(T) = -\sum_i P(T_i)\log_2 P(T_i) = -(\frac{1}{4}(-2 - 2 - 2) + \frac{1}{8}(-3) + \frac{1}{16}(-4 - 4)) = 2.375 \approx 3\]
		We see that the entropy gives an estimation of the least required length of encoding for a string.
	\end{itemize}
\section*{Problem 2}
\textbf{Solution:} 
	\begin{itemize}
		\item \textbf{Base cases:} When $n = 1$, the decision tree does not need to grow, thus $f(1) = 0$.
			  When $n = 2$, the decision needs to at least grow once and have two leaves, thus $f(2) = 2$.
			  When $n = 3$, one of the leaf in the case of $n = 2$ needs to be split, and we get that $f(3) = 2 + 2 + 1 = 5$.
			  
			  \textbf{Inductive hypothesis:} Suppose for all $4 \leq n \leq k \in \mathbb{N}$ there is 
			  \[f(n) = \min_{1\leq i\leq 3}[f(n - i) + f(i) + n]\]
			  
			  Then when $n =k + 1$, in order to accommodate the new data in tree, while introduce least new path length, we apply a strategy to split the lowest leave in tree of $n = k$, since decision tree is a binary tree, we can determine the lowest tree by
			  \begin{align}
			  	&h_{\text{lowest}}(n) = \lceil \log_2(n) \rceil\nonumber
			  \end{align}
			  
			  so that
			  \begin{align}
			  	&f(k + 1) = f(k) + 2(h_{\text{lowest}}(k) + 1) - h_{\text{lowest}}(k)\nonumber\\
			  	&\phantom{f(k + 1)} = f(k) + h_{\text{lowest}}(k) + 2\nonumber\\
			  	&\phantom{f(k + 1)} = \min_{1\leq i\leq 3} [f(k - i) + f(i) + k + h_{\text{lowest}}(k) + 2]\nonumber\\
			  	&\phantom{f(k + 1)} = \min_{1\leq i\leq 3} [f(k - i) + f(i) + k + h_{\text{lowest}}(k) + 2 + h_{\text{lowest}}(k - i) - h_{\text{lowest}}(k - i)]\nonumber\\
			  	&\phantom{f(k + 1)} = \min_{1\leq i\leq 3} [(f(k - i) + h_{\text{lowest}}(k - i) + 2) + f(i) + k + h_{\text{lowest}}(k) - h_{\text{lowest}}(k - i)]\nonumber\\
			  	&\phantom{f(k + 1)} = \min_{1\leq i\leq 3} [f(k + 1 - i) + f(i) + k + h_{\text{lowest}}(k) - h_{\text{lowest}}(k - i)]\nonumber\\
			  	&\phantom{f(k + 1)} = \min_{1\leq i\leq 3} [f(k + 1 - i) + f(i) + k + 1]\nonumber
			  \end{align}
			  
			  is proved.
			  
			  By Strong Induction, we conclude the formula holds for all $n \in \mathbb{N}$.
			  \item 
			  \textbf{Claim 1:} Let the total length of path of $D(T_d)$ be $L(D(T_d))$, then $f(|X'|) = L(D(T_d)) \Leftrightarrow$ $D$ is optimal.
			  
			  \textbf{Proof:} "$\Rightarrow$": Suppose $D$ is not optimal then there is $D'$ that $L(D') < L(D) = f(|X'|)$. This is impossible since $f$ is the minimum length of path a tree can have, thus we see that $D$ has to be optimal.
			  
			  "$\Leftarrow$": If $L(D) \neq f(|X'|)$, then $L(D) > f(|X'|)$ since $f$ is the minimum, then there is a smaller tree $D'$ that $L(D') < L(D)$, thus $D$ is not optimal. Hence, $L(D) = f(|X'|)$.
			  
			  We conclude that $f(|X'|) = L(D(T_d)) \Leftrightarrow$ $D$ is optimal.
			  
			  \textbf{Claim 2:} $f(|X'|) = L(D(T_d))$ if and only if  the set of non-singleton tests used in this optimal tree form an exact cover of $X$.
			  
			  \textbf{Proof:} "$\Rightarrow$": Suppose the non-singleton tests used in this optimal tree does not form an exact cover of $X$, then there are $t_1, t_2 \in \mathcal{T}'$ that $t_1 \cap t_2 \neq \emptyset$, thus at least one more split is needed to separate the overlapped symbols in two different subtrees, which cause the length of path larger than $f(|X'|)$. Thus, the non-singleton tests used in this optimal tree does form an exact cover of $X$.
			  
			  "$\Leftarrow$": If $L(D) \neq f(|X'|)$, then $L(D) > f(|X'|)$ since $f$ is the minimum, then there are tests $t_1, t_2 \in \mathcal{T}'$ that $t_1 \cap t_2 \neq \emptyset$, the non-singleton tests used in this optimal tree does not form an exact cover of $X$. Hence, $L(D) = f(|X'|)$.
			  
			  So we conclude that $f(|X'|) = L(D(T_d))$ if and only if  the set of non-singleton tests used in this optimal tree form an exact cover of $X$.
			  	
			  \item Just as the question mentions, we define $DT(\mathcal{T}', X', w)$ where $X' = X \cup \{a, b, c\}$, $\{a, b, c\} \cap \emptyset$, $\mathcal{T}' = \mathcal{T} \cup \{\{x\} \mid x \in X'\}$, $w = f(|X'|)$. As we proved in step 2, $DT(\mathcal{T}', X', w)$ is true if and only if $EC3(\mathcal{T}, X)$. Since $EC3$ is NP-hard and $EC3$ can be reduced to decision tree problem, we see that DT is also NP-hard.
			  \item By having the minimum number of tests, each time we need to predict from a set of features, minimum amount of calculation is needed and increased the speed as a result. Also, by having minimum number of tests, variance can be reduced.
	\end{itemize}
\newpage \nocite{*}
\bibliographystyle{ims}
\bibliography{citations}

\end{document}
