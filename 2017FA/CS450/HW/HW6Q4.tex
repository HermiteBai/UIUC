\documentclass[11pt]{article}
%-----------Packeges---------------%
\usepackage{amsmath}
\usepackage{amssymb}
\usepackage{amsfonts}
\usepackage{tocloft}
\usepackage{float}
\usepackage{graphicx}
\usepackage[bookmarks=true]{hyperref}
\usepackage{fancyhdr}


%----------Definition & Theorem----%
\newtheorem{definition}{Definition}[subsection]
\newtheorem{theorem}{Theorem}[subsection]
\newtheorem{proposition}{Proposition}[subsection]
\newtheorem{lemma}{Lemma}[subsection]
\newtheorem{corollary}{Corollary}[subsection]

\pagestyle{fancy}
\fancyhead[L]{CS 450}
\fancyhead[C]{HW6Q4}
\fancyhead[R]{Lanxiao Bai}

\usepackage{listings}
\usepackage{color}

\definecolor{dkgreen}{rgb}{0,0.6,0}
\definecolor{gray}{rgb}{0.5,0.5,0.5}
\definecolor{mauve}{rgb}{0.58,0,0.82}
\allowdisplaybreaks[1]

\lstset{frame=tb,
  language=Python,
  aboveskip=3mm,
  belowskip=3mm,
  showstringspaces=false,
  columns=flexible,
  basicstyle={\small\ttfamily},
  numbers=none,
  numberstyle=\tiny\color{gray},
  keywordstyle=\color{blue},
  commentstyle=\color{dkgreen},
  stringstyle=\color{mauve},
  breaklines=true,
  breakatwhitespace=true,
  tabsize=3
}

\begin{document}
	\begin{itemize}
		\item The order of convergence of this method with respect to $\Delta x$ is $2$ because the data of $(\Delta x_i, Error_i)$ fits to the quadratic curve.
		\item We see that the solution for $CFL = 0.75$ has a quickly vibrating part around $x = 0$, while the solution for $CFL = 0.7$ perfectly matches the analytical solution. Since analytically AB3 requires $CFL < .7236$ to be stable, it is predictable that solution for $CFL = 0.75$ will be unstable.
	\end{itemize}
\end{document}
