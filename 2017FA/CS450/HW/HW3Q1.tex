\documentclass[11pt]{article}
%-----------Packeges---------------%
\usepackage{amsmath}
\usepackage{amssymb}
\usepackage{amsfonts}
\usepackage{tocloft}
\usepackage{float}
\usepackage{graphicx}
\usepackage[bookmarks=true]{hyperref}
\usepackage{fancyhdr}


%----------Definition & Theorem----%
\newtheorem{definition}{Definition}[subsection]
\newtheorem{theorem}{Theorem}[subsection]
\newtheorem{proposition}{Proposition}[subsection]
\newtheorem{lemma}{Lemma}[subsection]
\newtheorem{corollary}{Corollary}[subsection]

\pagestyle{fancy}
\fancyhead[L]{CS 450}
\fancyhead[C]{HW3Q1}
\fancyhead[R]{Lanxiao Bai}

\usepackage{listings}
\usepackage{color}

\definecolor{dkgreen}{rgb}{0,0.6,0}
\definecolor{gray}{rgb}{0.5,0.5,0.5}
\definecolor{mauve}{rgb}{0.58,0,0.82}
\allowdisplaybreaks[1]

\lstset{frame=tb,
  language=Python,
  aboveskip=3mm,
  belowskip=3mm,
  showstringspaces=false,
  columns=flexible,
  basicstyle={\small\ttfamily},
  numbers=none,
  numberstyle=\tiny\color{gray},
  keywordstyle=\color{blue},
  commentstyle=\color{dkgreen},
  stringstyle=\color{mauve},
  breaklines=true,
  breakatwhitespace=true,
  tabsize=3
}

\begin{document}
	\begin{enumerate}
		\item The characteristic polynomial of $A$ is 
		\begin{align}
			&\phantom{\Leftrightarrow\ \ }P(\lambda) = det(A - \lambda I)\nonumber\\
			&\Leftrightarrow P(\lambda) = (1 - \lambda)^2 - 4\nonumber\\
			&\Leftrightarrow P(\lambda) = \lambda^2 - 2\lambda - 3
		\end{align}
		\item By solving characteristic polynomial (1), we get
		\begin{align}
			&\lambda^2 - 2\lambda - 3 = (\lambda - 3)(\lambda + 1) = 0\nonumber
		\end{align}
		 so
		 \[\lambda_1 = 3, \lambda_2 = -1\]
		 \item As mentioned above, the eigenvalues is
		 \[\lambda_1 = 3, \lambda_2 = -1\]
		 \item For $\lambda_1 = 3$,
		 \begin{align}
		 	&\phantom{\Leftrightarrow\ \ }\begin{bmatrix}
		 		1 & 4\\
		 		1 & 1
		 	\end{bmatrix}
		 	x = 3x\nonumber\\
		 	&\Leftrightarrow\begin{bmatrix}
		 		1 & 4\\
		 		1 & 1
		 	\end{bmatrix}
		 	\begin{bmatrix}
		 		x_1\\
		 		x_2
		 	\end{bmatrix} = \begin{bmatrix}
		 		3x_1\\
		 		3x_2
		 	\end{bmatrix}\nonumber\\
		 	&\Leftrightarrow
		 	\begin{bmatrix}
		 		x_1 + 4x_2\\
		 		x_1 + x_2
		 	\end{bmatrix} = 
		 	\begin{bmatrix}
		 		3x_1\\
		 		3x_2
		 	\end{bmatrix}\nonumber\\
		 	&\Leftrightarrow
		 	\begin{cases}
		 		x_1 + 4x_2 = 3x_1\\
		 		x_1 + x_2 = 3x_2
		 	\end{cases}\nonumber\\
		 	&\Leftrightarrow 
		 	\begin{cases}
		 		4x_2 = 2x_1\\
		 		x_1 = 2x_2
		 	\end{cases}\nonumber\\
		 	&\Leftrightarrow 
		 	\mathbf{x}_1 = \begin{bmatrix}
		 		2c\\
		 		c
		 	\end{bmatrix} (c \in \mathbb{R})
		 \end{align}
		 
		 For $\lambda_2 = -1$,
		 \begin{align}
		 	&\phantom{\Leftrightarrow\ \ }\begin{bmatrix}
		 		1 & 4\\
		 		1 & 1
		 	\end{bmatrix}
		 	x = -x\nonumber\\
		 	&\Leftrightarrow\begin{bmatrix}
		 		1 & 4\\
		 		1 & 1
		 	\end{bmatrix}
		 	\begin{bmatrix}
		 		x_1\\
		 		x_2
		 	\end{bmatrix} = \begin{bmatrix}
		 		-x_1\\
		 		-x_2
		 	\end{bmatrix}\nonumber\\
		 	&\Leftrightarrow
		 	\begin{bmatrix}
		 		x_1 + 4x_2\\
		 		x_1 + x_2
		 	\end{bmatrix} = 
		 	\begin{bmatrix}
		 		-x_1\\
		 		-x_2
		 	\end{bmatrix}\nonumber\\
		 	&\Leftrightarrow
		 	\begin{cases}
		 		2x_1 = -4x_2\\
		 		x_1 = -2x_2
		 	\end{cases}\nonumber\\
		 	&\Leftrightarrow
		 	\mathbf{x}_2 = \begin{bmatrix}
		 		-2c\\
		 		c
		 	\end{bmatrix}(c \in \mathbb{R})
		 \end{align}
		 \item By applying power iteration once, we get
		 	\[x' = Ax / ||Ax|| = [5/\sqrt{29}, 2/\sqrt{29}]^T\]
		 \item By the result, of question 4 and 5, we see that eigenvector will converge to $[2/\sqrt{5}, 1/\sqrt{5}]^T$.
		 \item By using Rayleigh quotient, we get that 
		 \[\lambda = \frac{x^TAx}{x^Tx} = 3.5\]
		 
		 \item The inverse iteration will converge to the least eigenvalue in magnitude, which is $-1$.
		 \item Since power iteration converges to the closest eigenvalue to the shift, it will converges to $\lambda = 3$ when $\sigma = 2$.
		 \item Since $A$ is not symmetric, so it will converges to triangular matrix after QR iteration.
	\end{enumerate}
\end{document}
