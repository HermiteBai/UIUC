\documentclass[11pt]{article}
%-----------Packeges---------------%
\usepackage{amsmath}
\usepackage{amssymb}
\usepackage{amsfonts}
\usepackage{tocloft}
\usepackage{float}
\usepackage{graphicx}
\usepackage[bookmarks=true]{hyperref}
\usepackage{fancyhdr}


%----------Definition & Theorem----%
\newtheorem{definition}{Definition}[subsection]
\newtheorem{theorem}{Theorem}[subsection]
\newtheorem{proposition}{Proposition}[subsection]
\newtheorem{lemma}{Lemma}[subsection]
\newtheorem{corollary}{Corollary}[subsection]

\usepackage{enumerate}
\usepackage{stmaryrd}
\pagestyle{fancy}
\fancyhead[L]{Math444}
\fancyhead[C]{HW8}
\fancyhead[R]{Lanxiao Bai}
\newcommand{\qed}{
	\begin{flushright}
		$\blacksquare$
	\end{flushright}}
\begin{document}
	\paragraph{5.2.1}
		\begin{enumerate}[(a)]
			\item Since $x^2 \geq 0 \Rightarrow x^2 + 1 \geq 1 > 0$ for all $x \in \mathbb{R}$. And since $x^2 + 2x + 1$ is continuous on $\mathbb{R}$ by Example 5.2.3, so $f(x)$ is continuous on $\mathbb{R}$ by Theorem 5.2.2(b).
			\item Since $x$ is continuous on $\mathbb{R^*}$, $\sqrt{x}$ is continuous on $\mathbb{R^*}$ by Theorem 5.2.5(b). So $x + \sqrt{x}$ is continuous on $\mathbb{R^*}$ by Theorem 5.2.2(a), so $g(x)$ is continuous on $x \geq 0$ by Theorem 5.2.5(b).
			\item Since $x \neq 0$ is continuous, $\sin x$ is continuous, so $|\sin x|$ is continuous by Theorem 5.2.4 and $\sqrt{1 + |\sin x|}$ is continuous by Theorem 5.2.2 and Theorem 5.2.5. As a result, by Theorem 5.2.2(b) $h(x)$ is continuous when $x \neq 0$. 
			\item Since $x$ continuous on $\mathbb{R}$, $1 + x^2$ is continuous by Theorem 5.2.2(b), and since $\cos x$ is continuous on $\mathbb{R}$, so $k(x)$ is also continuous on $\mathbb{R}$ by Theorem 5.2.7.
		\end{enumerate}
	\paragraph{5.2.3}
		\begin{enumerate}[(a)]
			\item Let 
				\[f(x) = \begin{cases}
					x & \text{if } x \neq 0\\
					1 &  \text{otherwise}
				\end{cases} \]
				and
				\[g(x) = \begin{cases}
					x & \text{if } x \neq 0\\
					-1 &  \text{otherwise}
				\end{cases} \]
				
				Then $f, g$ are discontinuous at $x = 0$, but $f + g = 2x$ and is continuous on $\mathbb{R}$.
			\item Let 
				\[f(x) = \begin{cases}
					x & \text{if } x \neq 1\\
					2 &  \text{otherwise}
				\end{cases} \]
				and
				\[g(x) = \begin{cases}
					x & \text{if } x \neq 1\\
					1/2 &  \text{otherwise}
				\end{cases} \]
				
				Then $f, g$ are discontinuous at $x = 1$, but $fg = x^2$ and is continuous on $\mathbb{R}$.
		\end{enumerate}
	\paragraph{5.2.4}
		When $x \in \mathbb{Z}$, $\llbracket x\rrbracket = x + 1$, so $x - \llbracket x\rrbracket = -1$, and when $x \notin \mathbb{Z}$, $x - \llbracket x\rrbracket$ is the decimal part of $x$. As a result, $f(x)$ is discontinuous when $x \in \mathbb{Z}$ and continuous otherwise.
	\paragraph{5.2.5}
		By substitution, we see that $g \circ f(0) = g(1) = 0$ and from the definition of $f$ and $g$, we see that 
			\[g \circ f(x) = \begin{cases}
				2 & x \neq 0\\
				0 & x = 0
			\end{cases} \] 
			
		Thus, $\lim_{x \rightarrow 0} g \circ f(x) = 2 \neq 0 = g \circ f(0)$.
	\paragraph{5.2.9}\textbf{Proof:} Suppose that there is a $c \in \mathbb{R}$ that $h(c) \neq 0$. We notice that for any $m \in \mathbb{Z}$, $(m/2^n)$ converges to $0$, so we can always find a $\delta$ that there is $x \in (c - \delta, c + \delta)$, so that
			\begin{align}
				|h(x) - 0| = |h(x)| > \varepsilon\nonumber
			\end{align}
			
		So we have $f(x)$ does not converge at $c$ and thus is discontinuous, which contradicts with our condition given.
			
		Hence, for all $x \in \mathbb{R}$, $h(x) = 0$.
		\qed
	\paragraph{5.3.1}\textbf{Proof:}
		Since $I$ is a closed and bounded interval and $f$ is continuous, then there is $x^* \in I$ that $f(x^*) \leq f(x)$ for all $f(x)$ by Maximum-Minimum Theorem.
		
		Since for all $x \in I$, $f(x^*) > 0$, then by Corollary 2.4.5, there is $n \in \mathbb{N}$ that $\alpha = 1/n$ have $0 < \alpha < f(x^*) \leq f(x)$ for all $x \in I$.
		\qed
	\paragraph{5.3.3}\textbf{Proof:} If for all $x \in I$, $f(x) \geq 0$, then we have $b_n$ that $f(b_n) = \frac{1}{2}f(b_{n - 1})$, by Archimedean Property, we have $\inf \{f(b_n)\} = 0$. Since $I$ is closed, bounded interval and $f$ is continuous, we have that there is $c \in I$ that $f(c) = 0$.
	
	Similarly, if for all $x \in I$, $f(x) \leq 0$, we have that there is $c \in I$ that $f(c) = 0$.
	
	If there is $a, b \in I$ that $f(a) \leq 0, f(b) \geq 0$, then by Location of Roots Theorem, there is a number $c \in (a, b) \subseteq I$ that $f(c) = 0$.
	
	In conclusion, there is a $c \in I$ that $f(c) = 0$.
	\qed
	\paragraph{5.3.4}\textbf{Proof:}
		Let polynomial of odd degrees be 
			\[f(x) = c_1x + c_2x^3 + \cdots + c_nx^{2n - 1}\]
			
		Since for all $n \in \mathbb{N}$, $x = 0 \Rightarrow x^n = 0 \Rightarrow cx^n = 0$ for any $c \in \mathbb{R}$. As a result, $f(0) = 0$.
		
		Hence, for all polynomials of odd degree, there is always a real root.
		\qed
	\paragraph{5.3.13}\textbf{Proof:}
		Since $f$ is bounded then there is $M$ that $|f(x)| \leq M$ for all $x \in \mathbb{R}$, then by Completeness Axiom, there is supremum and infimum of $f(x)$ in $\mathbb{R}$. Since $f(x)$ converges to $0$ when at $\infty$ and $-\infty$. 
		
		Then for any bounded closed interval $[a_n, b_n] \subseteq \mathbb{R}$, $f$ has both maximum $f(x^*_n)$ and minimum $f(x_{*n})$. Then for $\mathbb{R}$, we know that $\sup \{f(x)\} = \max \{x^*_1, x^*_2, \cdots, x^*_n, 0\}$ and $\min \{x_{*1}, x_{*2}, \cdots, x_{*n}, 0\}$. 
		
		Hence, either maximum or minimum can be attained.
		\qed
		
		And $y = x^3$ does not have any maximum or minimum.
	\paragraph{5.3.14}\textbf{Proof:}
		Suppose there is not a $\delta$-neignborhood of $x_0$ that make $f(x) < \beta$, then either $f(x)$ diverges, or it converges to $\alpha$ that $\alpha \geq \beta$.
		
		If $f(x)$ diverges, then $f$ is discontinuous at $x = x_0$, if $\lim_{x \rightarrow x_0} f(x) = \alpha \geq \beta$, then $\lim_{x \rightarrow x_0} f(x) \neq f(x_0)$, $f$ is still discontinuous, which contradicts with the condition given.
		
		Hence, there must be a $\delta$-neignborhood of $x_0$ that make $f(x) < \beta$.
		\qed
	\paragraph{5.4.2}\textbf{Proof:}
		Since $f(x) = 1/x^2$, then for all $\varepsilon > 0$ and $x, u \in [1, \infty)$, then when $|x - u| < \delta = \frac{\varepsilon x^2u^2}{x + u}$, we have 
		\[|f(x) - f(u)| = \frac{|u^2	 - x^2|}{x^2u^2} =  \frac{|u - x|(x + u)}{x^2u^2} < \frac{\delta(x + u)}{x^2u^2} = \varepsilon \]
		
		Hence, by definition, $f$ is uniformly continuous on $[1, \infty)$.
		
		However, when in $(0, \infty)$, let $(x_n) = \frac{1}{\sqrt{n}}$ and $(u_n) = \frac{1}{\sqrt{n + 1}}$, then $\lim (x_n - u_n) = 0$, but $|f(x_n) - f(u_n)| = 1$ for all $n \in \mathbb{N}$. 
		
		Hence, by Nonuniform Continuity Criteria, $f$ is not uniformly continuous on $(0, \infty)$.
		\qed 
	\paragraph{5.4.3}
		\begin{enumerate}[(a)]
			\item \textbf{Proof:}
					Suppose $\varepsilon = 1$, then whenever $|x - u| < \delta$,
					\[|x^2 - u^2| = |x^2 - (x + \delta)^2| = |2x\delta + \delta^2| \geq 1\]
					
					can be reached if we pick $x > \frac{1}{2\delta}$.
					
					Hence, by Nonuniform Continuity Criteria, $f$ is not uniformly continuous on $[0, \infty)$.
					\qed
			\item \textbf{Proof:}
				Let $(x_n) = \frac{1}{n}$ and $(u_n) = \frac{1}{n + \pi}$, then
				
				\[\lim(x_n - u_n) = \lim \frac{\pi}{n(n + \pi)} = 0\]
				
				but
				\[|f(x_n) - f(u_n)| = 2|\sin n|\]
				
				For $\varepsilon > 0$, there is always $n$ that $2|\sin n| \geq \varepsilon$.
				
				Hence, by Nonuniform Continuity Criteria, $f$ is not uniformly continuous on $(0, \infty)$.
				
		\end{enumerate}
	\paragraph{5.4.6}\textbf{Proof:}
		Since $f, g$ is uniformly continuous, then both $f, g$ are continuous on $A$ and whenever $x, u \in A$ has that $|x - u| < \delta$, there is $|f(x) - f(u)| < \varepsilon$ and $|g(x) - g(u)| < \varepsilon$ for all $\varepsilon > 0$ be definition.
		
		Thus, there is a $M > 0$ that $|f(x)| \leq M$ and $|g(x)| \leq M$. Then if we pick $\delta$ that $|f(x) - f(u)| < \varepsilon/2M$ and $|g(x) - g(u)| < \varepsilon/2M$.
		
		\begin{align}
			&|f(x)g(x) - f(u)g(u)| = |f(x)g(x) - f(x)g(u) + f(x)g(u) - f(u)g(u)|\nonumber\\
			&\phantom{|f(x)g(x) - f(u)g(u)|} \leq |f(x)(g(x) - g(u))| + |(f(x) - f(u))g(u)|\nonumber\\
			&\phantom{|f(x)g(x) - f(u)g(u)|} = |f(x)||g(x) - g(u)| + |f(x) - f(u)||g(u)|\nonumber\\
			&\phantom{|f(x)g(x) - f(u)g(u)|} \leq M(|g(x) - g(u)| + |f(x) - f(u)|) < \varepsilon\nonumber
		\end{align}
		
		Hence, by definition, $fg$ is uniformly continuous on $A$.
		\qed
	\paragraph{5.4.7}\textbf{Proof:}
		For $f(x) = x$, since $|f(u) - f(x)| = |u - x|$, so $f$ us a Lipschitz function, so $f$ is uniformly continuous on $\mathbb{R}$ by Theorem 5.4.5.
		
		For $g(x) = \sin x$, $|g(u) - g(x)| = |2\sin(x/2 - u/2)\cos(x/2 + u/2)| \leq |2\sin(x/2 - u/2)|$. If we pick $\delta = 2\arcsin(\varepsilon)$, then there is $|g(u) - g(x)| \leq \varepsilon$. Thus, by definition, $g$ is uniformly continuous on $\mathbb{R}$.
		
		Then for $fg(x) = x\sin x$, suppose it is uniformly continuous, then for $\varepsilon > 0$, if $|x - u| < \delta$, $|f(x) - f(u)| < \varepsilon$, but if we let $p = x + 2n\pi, q = u + 2n\pi$, then when $|p - q| < \delta$, but $|f(p) - f(u)| = |(x\sin(x)-y\sin(y))+2n\pi(\sin(x)-\sin(y))|>\varepsilon$ when $n$ is sufficiently large.
		
		Hence, $fg$ is not uniformly continuous on $\mathbb{R}$.
		\qed
\end{document}
