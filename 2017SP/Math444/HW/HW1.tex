\documentclass[11pt]{article}
%-----------Packeges---------------%
\usepackage{amsmath}
\usepackage{amssymb}
\usepackage{amsfonts}
\usepackage{tocloft}
\usepackage{float}
\usepackage{graphicx}
\usepackage[bookmarks=true]{hyperref}
\usepackage{fancyhdr}


%----------Definition & Theorem----%
\newtheorem{definition}{Definition}[subsection]
\newtheorem{theorem}{Theorem}[subsection]
\newtheorem{proposition}{Proposition}[subsection]
\newtheorem{lemma}{Lemma}[subsection]
\newtheorem{corollary}{Corollary}[subsection]

\pagestyle{fancy}
\fancyhead[L]{Math 444}
\fancyhead[C]{HW 1}
\fancyhead[R]{Lanxiao Bai(lbai5)}
\begin{document}

\paragraph{1.1.7}\textbf{Solution:}
	\subparagraph{(a)}
	 Claim: $A_1 \cap A_2 = \{6k: k \in \mathbb{N}\}$.
	 
	 Proof:
		Since $\forall n \in \mathbb{N}, A_n = \{(n + 1)k: k \in \mathbb{N}\}$, we have $A_1 = \{2k: k \in \mathbb{N}\}$ and $A_2 = \{3k: k \in \mathbb{N}\}$. So $\forall x \in A_1 \cap A_2, \exists k_1, k_2$ that $x = 2k_1 = 3k_2$, so $2 | x \Rightarrow 2 | 3k_2 \Rightarrow 2 | k_2 \Rightarrow \exists k_3$ that $k_2 = 2k_3$, so $x = 6k_3, k \in \mathbb{N}$, which means $A_1 \cap A_2 \subseteq \{6k: k \in \mathbb{N}\}$.
		
		On the other hand, $\forall x \in \{6k: k \in \mathbb{N}\}, \exists k_0$ that $x = 6k_0$, so naturally, we have $k_1 = 3k_0, k_2 = 2k_1$ that $x = 2k_1 = 3k_2$, so $\{6k: k \in \mathbb{N}\} \subseteq A_1 \cap A_2$.
		
		As a result, $A_1 \cap A_2 = \{6k: k \in \mathbb{N}\}$.$\blacksquare$
	\subparagraph{(b)}
	$\bigcup\{A_n: n \in \mathbb{N}\} = \mathbb{N} - \{1\}$, $\bigcap\{A_n: n \in \mathbb{N}\} = \emptyset$.
\paragraph{1.1.10}\textbf{Solution:}
	\subparagraph{(a)}
		Since $E := \{x \in \mathbb{R}: 1 \leq x \leq 2\}$, $x^2 \in [1, 4]$, so $f(E) = \{x \in \mathbb{R}: 1/4 \leq x \leq 1\}$.
	\subparagraph{(b)}
		Since $G := \{x \in \mathbb{R}: 1 \leq x \leq 4\}, f^{-1}(G) = \{x \in \mathbb{R}: 1/2 \leq x \leq 1\}$
\paragraph{1.1.14}\textbf{Solution:}
	Proof: 
		Prove $f(E \cup F) = f(E) \cup f(F)$ first.
		
		$\forall x \in E \cup F$, $x \in E$ or $x \in F$, so $f(x) \in f(E)$ or $f(x) \in f(F)$. That is saying that $f(x) \in f(E) \cup f(F)$, so $f(E \cup F) = f(E) \cup f(F)$.
		
		$\forall f(x) \in f(E) \cup f(F)$, then $\exists x \in E$ or $x \in F$, which means $x \in E \cup F$, so $f(x) \in f(E \cup F)$. Thus, $f(E) \cup f(F) \subseteq f(E \cup F)$.
		
		As a result, $f(E \cup F) = f(E) \cup f(F)$.
		
		Then $\forall x \in E \cap F$, $x \in E$ and $x \in F$, so $f(x) \in f(E)$ and $f(x) \in f(F)$ which means $f(x) \in f(E) \cap f(F)$.
		
		As a result, $f(E \cap F) \subseteq f(E) \cap f(F)$.$\blacksquare$
\paragraph{1.1.16}\textbf{Solution:}
	Proof:
	$\forall x_1, x_2 \in \mathbb{R}$,  if $f(x_1) \neq f(x_2)$, $x_1/\sqrt{x^2_1 + 1} \neq x_2/\sqrt{x^2_2 + 1} \Rightarrow x_1\sqrt{x^2_2 + 1} \neq x_2\sqrt{x^2_1 + 1} \Rightarrow x_1^2x_2^2 + x_1^2 \neq x_1^2x_2^2 + x_2^2 \Rightarrow x_1^2 \neq x_2^2 \Rightarrow x_1 \neq x_2$, which proves that $f$ is injective.
	
	$\forall y \in B$, $y = x/\sqrt{x^2 + 1} \Rightarrow y^2 = x^2/(x^2 + 1) \Rightarrow x^2(y^2 - 1) = -y^2 \Rightarrow x^2 = y^2/(1 - y^2) \Rightarrow x = y/\sqrt{1 - y^2}$ exists since $y \in B$ and $y^2 < 1$ as a result.
	
	So $f$ is a bijection.$\blacksquare$

\paragraph{1.2.2}\textbf{Solution:}
	Proof:
	
	Base case: when $n = 1$, $1^3 = 1 = [\frac{1}{2}1(1 + 1)]^2$.
	
	Suppose when $n = k$, we have
	\[1^3 + 2^3 + \cdot + k^3 = [\frac{1}{2}k(k + 1)]^2\]
	
	Then when $n = k + 1$,
	
	\begin{align}
		&1^3 + 2^3 + \cdot + k^3 + (k + 1)^3 = [\frac{1}{2}k(k + 1)]^2 + (k + 1)^3\nonumber\\
		&\phantom{1^3 + 2^3 + \cdot + k^3 + (k + 1)^3} = \frac{1}{4}k^4 + \frac{1}{2}k^3 + \frac{1}{4}k^2 + k^3 + 3 k^2 + 3 k + 1\nonumber\\
		&\phantom{1^3 + 2^3 + \cdot + k^3 + (k + 1)^3} = \frac{1}{4}(k^4 + 6k^3 + 26k^2 + 3k + 1)\nonumber\\
		&\phantom{1^3 + 2^3 + \cdot + k^3 + (k + 1)^3} = \frac{1}{4} (k + 1)^2 (k + 2)^2\nonumber\\
		&\phantom{1^3 + 2^3 + \cdot + k^3 + (k + 1)^3} = [\frac{1}{2}(k + 2)(k + 1)]^2\nonumber
	\end{align} 
	
	So by mathematical induction, we proved that $1^3 + 2^3 + \cdot + n^3 = [\frac{1}{2}n(n + 1)]^2$, $\forall n \in \mathbb{N}$.$\blacksquare$

\paragraph{1.2.6}\textbf{Solution:}
	Proof:
	
	Base case: when $n = 1$, $n^3 + 5n = 6$ which is divisible by $6$.
	
	Suppose that when $n = k$, we have $k^3 + 5k$ is divisible by $6$. Then when $n = k + 1$, $(k + 1)^3 + 5(k + 1) = k^3 + 3 k^2 + 5 k + 3k + 6 = (k^3 + 5k) + (3k^2 + 3k + 6) = 6(k_0 + 1) + 3k(k + 1)$. Since one of $k$ and $k + 1$ have to be even, so $3k(k + 1)$ is divisible by $6$. Thus $(k + 1)^3 + 5(k + 1) = 6(k_0 + 1) + 3k(k + 1)$ is divisible by $6$.
	
	As a result, $n^3 + 5n$ is divisible by $6$ for all $n \in \mathbb{N}$.$\blacksquare$
\paragraph{1.2.13}\textbf{Solution:}
	Proof:
	
	Base case: when $n = 1$, $1 < 2^1 = 2$.
	
	Suppose when $n = k$, we have $k < 2^k$. Then when $n = k + 1$, $k + 1 < 2^k + 1 < 2^k + 2^1 < 2^k + 2^k < 2^{k + 1}$.
	
	As a result, $n < 2^n$ for all $n \in \mathbb{N}$.
	
\paragraph{1.3.2}\textbf{Solution:}
	\subparagraph{(b)} Proof:
		Since $|A| = m$, $|C| = 1$ and $C \subset A$, $|A \cap C| = |C| = 1$, $|A/C| = |A| - |A \cap C| = m - 1$.$\blacksquare$
	\subparagraph{(c)} Proof:
		Suppose $C\slash B$ is a finite set, either $|C\slash B| = 0$ or $n \in \mathbb{N}$ that $|C\slash B| = n$. Without generality, we can suppose $(B \slash C) = m(0 \leq m \leq |B|)$. Then $|C| = |B| - |B \slash C| + |C \slash B| = |B| - m + n$, which means $C$ is finite. That is not possible according to the condition given.
		
		Thus, $C\slash B$ is infinite.$\blacksquare$ 
\paragraph{1.3.4}\textbf{Solution:}
	Claim: Let $f: \mathbb{N} \rightarrow (13, \infty)$, $f(x) = x + 13$, it is a bijection.
	
	Proof: $\forall x_1, x_2 \in \mathbb{N}$, $f(x_1) \neq f(x_2) \Rightarrow x_1 + 13 \neq x_2 + 13 \Rightarrow x_1 \neq x_2$, which means $f$ is injective.
	
	And $\forall y \in (13, \infty)$, $\exists x = y - 13 \in \mathbb{N}$ that $y = x + 13$. So $f$ is surjective.
	
	As a result, $f$ is a bijection.$\blacksquare$
		
\paragraph{1.3.5}\textbf{Solution:}
	\[
		f(x) = 
		\begin{cases}
			x / 2 \mathrm{\ if\ }x\mathrm{\ is\ even}\\
			-(x + 1) / 2 \mathrm{\ if\ }x\mathrm{\ is\ odd}
		\end{cases}
	\]
\end{document}
