\documentclass[11pt]{article}
%-----------Packeges---------------%
\usepackage{amsmath}
\usepackage{amssymb}
\usepackage{amsfonts}
\usepackage{tocloft}
\usepackage{float}
\usepackage{graphicx}
\usepackage[bookmarks=true]{hyperref}
\usepackage{fancyhdr}


%----------Definition & Theorem----%
\newtheorem{definition}{Definition}[subsection]
\newtheorem{theorem}{Theorem}[subsection]
\newtheorem{proposition}{Proposition}[subsection]
\newtheorem{lemma}{Lemma}[subsection]
\newtheorem{corollary}{Corollary}[subsection]

\usepackage{enumerate}
\usepackage{stmaryrd}
\pagestyle{fancy}
\fancyhead[L]{Math444}
\fancyhead[C]{HW9}
\fancyhead[R]{Lanxiao Bai}
\newcommand{\qed}{
	\begin{flushright}
		$\blacksquare$
	\end{flushright}}
\begin{document}
	\paragraph{6.1.2}\textbf{Proof:}
		Suppose $f(x) = x^{1/3}$ is differentiable at $x = 0$, then given $\varepsilon > 0$, there is $\delta > 0$ that when $0 < |x - 0| = |x| < \delta$, then
			\[\left|\frac{f(x) - f(0)}{x - 0}\right| < \varepsilon\]
			
		so
			\[\left|\frac{x^{1/3}}{x}\right| = x^{-2/3} < \varepsilon\]
			
		However, when $\varepsilon = 1$, for all $0 < x < 1$, whenever $0 < |x| < \delta$, there is $|x^{-2/3}| > 1 = \varepsilon$, which leads to a contradiction.
		
		Hence, $f(x) = x^{1/3}$ is not differentiable at $x = 0$.
		\qed
	\paragraph{6.1.4}\textbf{Proof:} Given that $\varepsilon > 0$, if $\delta = \varepsilon$, then when $0 < |x - 0| < \delta = \varepsilon$, we have
		\begin{align}
			&\left|\frac{f(x) - f(0)}{x - 0}\right| = \left|\frac{f(x)}{x}\right|\nonumber\\
			&\phantom{\left|\frac{f(x) - f(0)}{x - 0}\right|} = \left|\frac{f(x)}{x}\right|\nonumber\\
			&\phantom{\left|\frac{f(x) - f(0)}{x - 0}\right|} = \begin{cases}
				|x| &\text{if $x$ is rational}\\
				0 &\text{otherwise}
				\end{cases}\nonumber\\
			&\phantom{\left|\frac{f(x) - f(0)}{x - 0}\right|} < \delta = \varepsilon\nonumber
		\end{align}
		
		Hence, by definition $f$ is differentiable at $x = 0$ and $f'(0) = 0$.
		\qed
	\paragraph{6.1.7}\textbf{Proof:}
		\paragraph{``$\Rightarrow$":}
			Since $g(x)$ is differentiable, then by definition, for all $\varepsilon > 0$, if we pick $\delta > 0$ that when $0 < |x - c| < \delta$, then
			\begin{align}
				&\left|\frac{g(x) - g(c)}{x - c} - L\right| = \left|\frac{|f(x)| - |f(c)|}{x - c} - L\right|\nonumber\\
				&\phantom{\left|\frac{g(x) - g(c)}{x - c} - L\right|} = \left|\frac{|f(x)|}{x - c} - L\right| \leq \left|\frac{|f(x)|}{x - c}\right| + |L|\nonumber\\
				&\phantom{\left|\frac{g(x) - g(c)}{x - c} - L\right|} =\left|\frac{f(x)}{x - c}\right| + |L|\nonumber\\
				&\phantom{\left|\frac{g(x) - g(c)}{x - c} - L\right|} < \varepsilon + |L|\nonumber
			\end{align}
			
			Then
			\[\left|\frac{f(x) - f(c)}{x - c} - 0\right| < \varepsilon\]
			
			Hence, by definition, $g(x) = |f(x)|$ is differentiable at $x = c$ $\Rightarrow$ $f(x)$ is differentiable and $f'(c) = 0$.
			
		\paragraph{``$\Leftarrow$":}
			Since $f'(c) = 0$, then for all $\varepsilon > 0$ if we pick $\delta$ that when $x \in (c - \delta, c + \delta)$, there is
			\begin{align}
				&\phantom{\Rightarrow}\left|\frac{f(x) - f(c)}{x - c} - 0\right| = \left|\frac{f(x)}{x - c}\right| < \varepsilon\nonumber\\
				&\Rightarrow |f(x)| < \varepsilon|x - c| < \varepsilon\delta - |L|\nonumber
			\end{align} 
			
			As a result, for any $L \geq 0$, 
			\begin{align}
				&\left|\frac{g(x) - g(c)}{x - c} - L\right| = \left|\frac{|f(x)|}{x - c} - L\right|\nonumber\\
				&\phantom{\phantom{\left|\frac{f(x) - f(c)}{x - c} - 0\right|}} \leq \frac{|f(x)|}{|x - c|} + |L|\nonumber\\
				&\phantom{\phantom{\left|\frac{f(x) - f(c)}{x - c} - 0\right|}} < \varepsilon\nonumber
			\end{align}
			
			Hence by definition, $f(x)$ is differentiable at $c$ and $f'(c) = 0$ $\Rightarrow$ $g(x) = |f(x)|$ is differentiable at $x = c$.
	\paragraph{6.2.3}
		\begin{enumerate}[(a)]
			\item Since $f(x) = x + 1/x$, $f'(x) = 1 - 1/x^2$, so that let $f'(x) = 0 \Rightarrow x = 1$ or $x = -1$, which are local extremums. Since $f(0.5) = 2.5 > 2 = f(1)$ and $f(2) = 2.5 > 2 = f(1)$, $f(-0.5) = -2.5 < -2 = f(-1)$ and $f(-2) = -2.5 < -2 = f(-1)$. We know that relative maximum is $(-1, -2)$ and relative minimum is $(1, 2)$ and the increasing intervals are $(-\infty, -1] \cup [1, \infty)$ and the decreasing intervals are $(-1, 0) \cup (0, 1)$.
			\item Since $g(x) = x/(x^2 + 1$, $g'(x) = (1 - x^2)/(x^2 + 1)^2$, so that let $f'(x) = 0 \Rightarrow x = 1$ or $x = -1$, which are relative extremums. Since $g(0.5) = 0.4 < 0.5 = g(1)$ and $g(2) = 0.4 < 0.5 = g(1)$, $g(-0.5) = -0.5 < -0.4 = g(-1)$ and $g(-2) = -0.5 < -0.4 = g(-1)$. We know that relative maximum is $(1, 0.5)$ and relative minimum is $(-1, -0.5)$ and the decreasing intervals are $(-\infty, -1] \cup [1, \infty)$ and the increasing intervals are $(-1, 1)$.
			\item Since $h(x) = x^3 - 3x - 4$, $h'(x) = 3x^2 - 3$, so that let $h'(x) = 0 \Rightarrow x = 1$ or $x = -1$, which are relative extremums. Since $h(0) = -4 > -6 = h(1)$ and $h(2) = -2 > -6 = h(1)$, $h(0) = -4 < -2 = h(-1)$ and $h(-2) = -6 < -2 = h(-1)$. We know that relative maximum is $(-1, -2)$ and relative minimum is $(1, -6)$ and the increasing intervals are $(-\infty, -1] \cup [1, \infty)$ and the decreasing intervals are $(-1, 1)$.
			\item Since $k(x) = 2x + 1/x^2$, $g'(x) = 2 - 2/x^3$, so that let $k'(x) = 0 \Rightarrow x = 1$, which is relative extremums. Since $f(0.5) = 5 > 3 = f(1)$ and $f(2) = 4.25 > 3 = f(1)$, $f(-0.5) = -0.75 > -1 = f(-1)$. We know that relative minimum is $(1, 3)$ and the increasing intervals are $(-\infty, 0) \cup [1, \infty)$ and the decreasing intervals are $(0, 1)$.
		\end{enumerate}
\end{document}
