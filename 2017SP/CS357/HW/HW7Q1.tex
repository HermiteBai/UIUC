\documentclass[11pt]{article}
%-----------Packeges---------------%
\usepackage{amsmath}
\usepackage{amssymb}
\usepackage{amsfonts}
\usepackage{tocloft}
\usepackage{float}
\usepackage{graphicx}
\usepackage[bookmarks=true]{hyperref}
\usepackage{fancyhdr}


%----------Definition & Theorem----%
\newtheorem{definition}{Definition}[subsection]
\newtheorem{theorem}{Theorem}[subsection]
\newtheorem{proposition}{Proposition}[subsection]
\newtheorem{lemma}{Lemma}[subsection]
\newtheorem{corollary}{Corollary}[subsection]

\usepackage{enumerate}
\pagestyle{fancy}
\fancyhead[L]{CS357}
\fancyhead[C]{HW7Q1}
\fancyhead[R]{Lanxiao Bai}
\begin{document}
	\begin{enumerate}
		\item Let 
			\[A = \begin{bmatrix}
				a_{11} & a_{12} & \cdots & a_{1n}\\
				a_{21} & a_{22} & \cdots & a_{2n}\\
				\vdots & \vdots & \ddots & \vdots\\
				a_{n1} & a_{n2} & \cdots & a_{nn}
			\end{bmatrix}\]
			
			Then 
			\[AS = \begin{bmatrix}
				a_{11} & a_{12} & \cdots & a_{1n}\\
				a_{21} & a_{22} & \cdots & a_{2n}\\
				\vdots & \vdots & \ddots & \vdots\\
				a_{n1} & a_{n2} & \cdots & a_{nn}
			\end{bmatrix}\cdot\begin{bmatrix}
				0 & 1 & & 0\\
				  & 0 & \ddots & \\
				  &   & \ddots & 1\\
				1 &    &       & 0 
			\end{bmatrix} = \begin{bmatrix}
				a_{1n} & a_{11} & \cdots & a_{1(n - 1)}\\
				a_{2n} & a_{21} & \cdots & a_{2(n - 1)}\\
				\vdots & \vdots & \ddots & \vdots\\
				a_{nn} & a_{n2} & \cdots & a_{nn}
			\end{bmatrix}\]
			
			As shown, $S$ is a shift matrix since it can shift each column $1$ step to right. And 
			\[S\mathrm{x} = [x_{n - 1}, x_0, \cdots, x_{n - 2}]\]
			
			For the same reason, 
			\[S^k\mathrm{x} = [x_{n - k}, x_{n - k + 1}, \cdots, x_1, x_2, \cdots, x_{n - k - 1}]\]
			\item No, since all eigenvalues of $S$, $|\lambda_i| = 1$, power method may not converge.
			\item \textbf{Proof:} 
			
			\begin{align}
				&Sx = \lambda x\nonumber\\
				&(S - \lambda I) x = 0\nonumber\\
				&\begin{bmatrix}
				-\lambda & 1 & & 0\\
				  & -\lambda & \ddots & \\
				  &   & \ddots & 1\\
				1 &    &       & -\lambda
			\end{bmatrix} = 0\nonumber\\
				&det(\begin{bmatrix}
				-\lambda & 1 & & 0\\
				  & -\lambda & \ddots & \\
				  &   & \ddots & 1\\
				1 &    &       & -\lambda
			\end{bmatrix}) = 0\nonumber
			\end{align} 
			
			So that
			\begin{align}
				&\begin{bmatrix}
					1 & & 0\\
				  	-\lambda & \ddots & \\
				    & \ddots & 1
				\end{bmatrix} - \lambda \begin{bmatrix}
					-\lambda & & 0\\
				  			 & \ddots & \\
				    & \ddots & -\lambda
				\end{bmatrix} = 0\nonumber\\
				&\Rightarrow 1 \pm \lambda = 0 \Rightarrow |\lambda| = 1\nonumber
			\end{align} 
			\item Since $Sx = \lambda x$, $S^2x = \lambda_1 \lambda_2 x$ that $\lambda_1, \lambda_2$ is eigenvalues of $S$. 
			
				Since $\lambda_k = e^{i(2\pi k / n)} = \cos(2\pi k / n) + i\sin(2\pi k / n)$ and $\cos(2\pi\frac{n + k}{n}) = \cos(2\pi \frac{k}{n})$ and $\sin(2\pi\frac{n + k}{n}) = \sin(2\pi \frac{k}{n})$ when $k < n$.
			
			Thus, there are always $n$ unique eigenvalues. For the same reason, $S^4, S^{2^k}$ and $S^n$ all have $n$ unique eigenvalues.
	\end{enumerate}
\end{document}
