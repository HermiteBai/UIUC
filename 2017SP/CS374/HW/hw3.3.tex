% ---------
%  Compile with "pdflatex hw0".
% --------
%!TEX TS-program = pdflatex
%!TEX encoding = UTF-8 Unicode

\documentclass[11pt]{article}
\usepackage{jeffe,handout,graphicx}
\usepackage[utf8]{inputenc}		% Allow some non-ASCII Unicode in source

%  Redefine suits
\usepackage{pifont}
\def\Spade{\text{\ding{171}}}
\def\Heart{\text{\textcolor{Red}{\ding{170}}}}
\def\Diamond{\text{\textcolor{Red}{\ding{169}}}}
\def\Club{\text{\ding{168}}}

\def\Cdot{\mathbin{\text{\normalfont \textbullet}}}
\def\Sym#1{\textbf{\texttt{\color{BrickRed}#1}}}

\newcommand{\comp}[1]{#1_{\text{comp}}}
\newtheorem{claim}{Claim}
\def\To{\leadsto}
\def\Tostar{\mathrel{\To\!\!\!^*}}



% =====================================================
%   Define common stuff for solution headers
% =====================================================
\Class{CS/ECE 374}
\Semester{Spring 2017}
\Authors{2}
\AuthorOne{Renheng Ruan}{rruan2}
\AuthorTwo{Lanxiao Bai}{lbai5}

%\Section{}

% =====================================================
\begin{document}

% ---------------------------------------------------------


\HomeworkHeader{3}{3}	% homework number, problem number

\begin{quote}

Let $L = \{0^i1^j2^k \mid k = 2(i+j)\}$.
\begin{enumerate}[(a)]
	\item Prove that $L$ is context free by describing a grammar for $L$.
	\item Prove that your grammar is correct. You need to prove
	that if $L \subseteq L(G)$ and $L(G) \subseteq L$ where $G$ is your
	grammar from the previous part.
\end{enumerate}

\end{quote}
\hrule

\begin{solution}
	\mbox{}
\begin{enumerate}[(a)]
\item
$S\to \e \mid SSS \mid \Sym0 S \Sym{22} \mid \Sym1 S \Sym{22} \mid \Sym0 S \Sym1 S \Sym{2222}$

\item
We separately prove  $L \subseteq L(G)$ and $L(G) \subseteq L$ as follows:

\begin{claim}
	$L(G) \subseteq L$, that is, every string in $L(G)$ has exactly twice as many \Sym2s as the sum of \Sym0s and \Sym1s.
\end{claim}

\begin{proof}
	For any string $u$, let $\Delta(u) = \#(\Sym2, u) - 2(\#(\Sym0, u)+\#(\Sym1, u))$.  We need to prove that $\Delta(w)=0$ for every string $w\in L(G)$.
	
	Let $w$ be an arbitrary string in $L(G)$. Assume that $\Delta(x) = 0$ for every string $x\in L(G)$s. Consider the \emph{shortest} derivation of $w$, and assume $\Delta(x) = 0$ for every string $x\in L(G)$ such that $\abs{x}<\abs{w}$. There are five cases to consider, depending on the first production in the derivation of $w$.
	
	\begin{itemize}
		\item
		If $w = \e$, then $\#(\Sym2, w) = 2(\#(\Sym0, w)+\#(\Sym1, w)) = 0$ by definition, so $\Delta(w) = 0$.
		
		\item
		Suppose the derivation begins $S \To SSS \Tostar w$.  Then $w = xyz$ for some strings $x,y,z\in L(G)$, each of which can be derived with fewer than $k$ productions.   The inductive hypothesis implies $\Delta(x) =\Delta(y) = 0$.  It immediately follows that $\Delta(w)=0$.\footnote{Alternatively: Suppose the \emph{shortest} derivation of $w$ begins $S \To SSS \Tostar w$. Then $w = xy$ for some strings $x,y\in L(G)$.  Neither $x$ or $y$ can be empty, because otherwise we could shorten the derivation of $w$.  Thus, $x$ and $y$ are both shorter than $w$, so the induction hypothesis implies\dots.  We need some way to deal with the decompositions $w=\e\Cdot w$ and $w=w\Cdot\e$, which are both consistent with the production $S\to SSS$, without falling into an infinite loop.}
		
		\item
		Suppose the derivation begins $S \To \Sym0S\Sym{22} \Tostar w$.  Then $w = \Sym0x\Sym{22}$ for some string $x\in L(G)$.  The inductive hypothesis implies $\Delta(x) = 0$.  It immediately follows that $\Delta(w)=0$.
		
		\item
		Suppose the derivation begins $S \To \Sym1 S\Sym{00} \Tostar w$.  Then  $w = \Sym1 x \Sym{22}$ for some string $x\in L(G)$.  The inductive hypothesis implies $\Delta(x) = 0$.  It immediately follows that $\Delta(w)=0$.
		
		\item
		Suppose the derivation begins $S \To \Sym0 S\Sym1S \Sym{2222} \Tostar w$.  Then  $w = \Sym0 x\Sym1 y \Sym{2222}$ for some strings $x,y\in L(G)$.  The inductive hypothesis implies $\Delta(x) =\Delta(y) = 0$.  It immediately follows that $\Delta(w)=0$.
		
	\end{itemize}
	In all cases, we conclude that $\Delta(w)=0$, as required.
\end{proof}

\begin{claim}
	$L \subseteq L(G)$; that is, $G$ generates every binary string with exactly twice as many \Sym2s as the sum of \Sym0s and \Sym1s.
\end{claim}

\begin{proof}
	For any string $u$, let $\Delta(u) = 2\#(\Sym1, u) - (\#(\Sym2, u)-2\#(\Sym0, u))$.  For any string $u$ and any integer $0\le i\le \abs{u}$, let \EMPH{$u_i$} denote the $i$th symbol in $u$, and let~\EMPH{$u_{\le i}$} denote the prefix of $u$ of length $i$. 
	
	Let $w$ be an arbitrary binary string with twice as many \Sym2s as the sum of \Sym0s and \Sym1s.  Assume that $G$ generates every  binary string $x$ that is shorter than $w$ and has twice as many \Sym2s as the sum of \Sym0s and \Sym1s.  There are two cases to consider:
	
	\begin{itemize}
		\item
		If $w = \e$, then $\e\in L(G)$ because of the production $S\to \e$.
		
		\item
		Suppose $w$ is non-empty.  To simplify notation, let $\Delta_i = \Delta(w_{\le i})$ for every index~$i$, and observe that $\Delta_0 = \Delta_{\abs{w}} = 0$.  There are several subcases to consider:
		
		
		\begin{itemize}\parindent 1.5em
			\item
			Suppose $\Delta_i = 0$ for some index $0<i<\abs{w}$.  Then we can write $w = xyz$, where $x$, $y$ and $z$ are non-empty strings with $\Delta(x)=\Delta(y)=\Delta(z) = 0$.  The induction hypothesis implies that $x,y,z\in L(G)$, and thus the production rule $S\to SSS$ implies that $w\in L(G)$.
			
			\item
			Suppose $\Delta_i > 0$ for all $0<i<\abs{w}$.  Then $w$ must begin with \Sym0, since otherwise $\Delta_1 = -2$ or $\Delta_2 = -1$, and the last two symbol in $w$ must be \Sym{22}, since otherwise $\Delta_{\abs{w}-1} = -1$.  Thus, we can write $w = \Sym0x\Sym{22}$ for some binary string~$x$.  We easily observe that $\Delta(x)=0$, so the induction hypothesis implies $x\in L(G)$, and thus the production rule $S\to \Sym0S\Sym{22}$ implies $w\in L(G)$.
			
			\item
			Suppose $\Delta_i < 0$ for all $0<i<\abs{w}$.  A symmetric argument to the previous case implies  $w = \Sym1x\Sym{22}$ for some binary string $x$ with $\Delta(x)=0$.  The induction hypothesis implies $x\in L(G)$, and thus the production rule $S\to \Sym1 S\Sym{22}$ implies $w\in L(G)$.
			
			\item
			Finally, suppose none of the previous cases applies: $\Delta_i<0$ and $\Delta_j>0$ for some indices $i$ and $j$, but $\Delta_i \ne 0$ for all $0<i<\abs{w}$.
			
			Let $i$ be the smallest index such that $\Delta_i<0$.  Because $\Delta_j$ either increases by $1$ or decreases by $2$ when we increment $j$, for all indices $0<j<\abs{w}$, we must have $\Delta_j > 0$ if $j<i$ and $\Delta_j < 0$ if $j\ge i$.
			
			In other words, there is a \emph{unique} index $i$ such that $\Delta_{i-1}>0$ and $\Delta_i<0$.  In particular, we have $\Delta_1 > 0$ and $\Delta_{\abs{w}-1} < 0$.  Thus, we can write $w = \Sym0 x \Sym1 y\Sym{2222}$ for some binary strings $x$ and $y$, where $\abs{\Sym1 y \Sym{22}} = i$.
			
			We easily observe that $\Delta(x)=\Delta(y)=0$, so the inductive hypothesis implies $x, y\in L(G)$, and thus the production rule $S\to \Sym0 S \Sym1 S \Sym{2222}$ implies  $w\in L(G)$.
		\end{itemize}
	\end{itemize}
	In all cases, we conclude that $G$ generates $w$.        
\end{proof}

\end{enumerate}
Idea comes from the question 4 and use its form.
\end{solution}

\end{document}
