% ---------
%  Compile with "pdflatex hw0".
% --------
%!TEX TS-program = pdflatex
%!TEX encoding = UTF-8 Unicode

\documentclass[11pt]{article}
\usepackage{jeffe,handout,graphicx}
\usepackage[utf8]{inputenc}		% Allow some non-ASCII Unicode in source

%  Redefine suits
\usepackage{pifont}
\def\Spade{\text{\ding{171}}}
\def\Heart{\text{\textcolor{Red}{\ding{170}}}}
\def\Diamond{\text{\textcolor{Red}{\ding{169}}}}
\def\Club{\text{\ding{168}}}

\def\Cdot{\mathbin{\text{\normalfont \textbullet}}}
\def\Sym#1{\textbf{\texttt{\color{BrickRed}#1}}}

\newcommand{\comp}[1]{#1_{\text{comp}}}
\newtheorem{claim}{Claim}
\def\To{\leadsto}
\def\Tostar{\mathrel{\To\!\!\!^*}}



% =====================================================
%   Define common stuff for solution headers
% =====================================================
\Class{CS/ECE 374}
\Semester{Spring 2017}
\Authors{3}
\AuthorOne{Renheng Ruan}{rruan2}
\AuthorTwo{Lanxiao Bai}{lbai5}
\AuthorThree{Ho Yin Au}{hoyinau2}

%\Section{}

% =====================================================
\begin{document}

% ---------------------------------------------------------


\HomeworkHeader{6}{3}	% homework number, problem number

\begin{quote}

\begin{enumerate}
	Suppose we need to distribute a message to all the nodes in a
	rooted tree. Initially, only the root node knows the message. In a
	single round, any node that knows the message can forward it to at
	most one of its children. Design an algorithm to compute the minimum
	number of rounds required for the message to be delivered to all
	nodes in a given tree.   See figure below for an example. 
	Assume that the tree is binary (number of children is at most $2$).
\end{enumerate}

\end{quote}
\hrule

\begin{solution}
	\mbox{}
Let T be the input tree. For each node v in T, let $MinRounds(v)$ denote the minimum number if rounds required, after v learns the message, to inform every descendant of v. And we want to compute $MinRounds(root(T))$. The following recursive algorithm evaluates this function recursively, essentially by memoizing MinRounds(v) at the parent of v, in the temporary variables $l$ and $r$.
\begin{center}
	\begin{algorithm}			
		\textul{$\textsc{MinRounds}(\textit{v})$:}\+
		\\ if $\textit{v}$ is a leaf \+
		\\  	$\textit{return}$ $0$\-
		\\ else if $\textit{v.right} = \textit{NULL}$ \+
		\\ 		return $1$ + \textsc{MinRounds}($\textit{v.left}$)\-
		\\ else if $\textit{v.left} = \textit{NULL}$ \+
		\\ 		return 1 + \textsc{MinRounds}($\textit{v.right}$)\-
		\\ else \+
		\\ $\textit{l} \gets \textsc{MinRounds}(v.\textit{left})$
		\\ $\textit{r} \gets \textsc{MinRounds}(v.\textit{right})$
		\\ if $\textit{l} < \textit{r}$\+
		\\ 		return $r+1$\-
		\\ else if $\textit{l} > \textit{r}$\+
		\\ 		return $l+1$\-
		\\ else\+
		\\ 		return $l+2$\-
	\end{algorithm}
\end{center}
Run Time: O(n)\\

\paragraph{} The solution template is taken from https://courses.engr.illinois.edu/cs473/homework/hw1-sol.pdf.

\end{solution}

\end{document}
