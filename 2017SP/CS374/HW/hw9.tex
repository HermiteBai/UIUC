%!TEX TS-program = pdflatex
%!TEX encoding = UTF-8 Unicode

\documentclass[11pt]{article}
\usepackage{jeffe,handout,graphicx,mathtools}
\usepackage[utf8x]{inputenc}			% allow Unicode in .tex file
\usepackage{enumerate}
\usepackage{fourier-orns}

\def\Sym#1{\texttt{\upshape\textcolor{BrickRed}{#1}}}
\def\SymBlue#1{\texttt{\upshape\textcolor{RoyalBlue}{#1}}}
\def\SymGreen#1{\texttt{\upshape\textcolor{PineGreen}{#1}}}
\def\_#1{\SymBlue{\underline{\smash{\textbf{#1}}}}}
\def\X#1{\SymGreen{$\overline{\textbf{#1}}$}}
\def\u#1{\raise0.5ex\hbox{\textcolor{RoyalBlue}{#1}}}

\def\Cdot{\mathbin{\text{\normalfont \textbullet}}}

\newcommand{\IsSinL}{\text{IsStringInL}}

% =========================================================
\begin{document}

\headers{CS/ECE 374}{Homework 9 (due April 19)}{Spring 2017}

\thispagestyle{empty}

\begin{center}
\Large\textbf{CS/ECE 374 \,\decosix\,  Spring 2017}%
\\
\LARGE\textbf{\decothreeleft~ Homework 9 ~\decothreeright}%
\\[0.5ex]
\large Due Wednesday, April 19, 2017 at 10am
\end{center}

\bigskip
\hrule
\bigskip

\noindent
\textbf{Groups of up to three people can submit joint solutions.}  Each problem should be submitted by exactly one person, and the beginning of the homework should clearly state the Gradescope names and email addresses of each group member.  In addition, whoever submits the homework must tell Gradescope who their other group members are.
\bigskip
\hrule
\bigskip


\noindent
The following unnumbered problems are not for submission or grading. 
No solutions for them will be provided but you can discuss them on Piazza.
\begin{itemize}
\item  Let $R_1,R_2,\ldots,R_n$ be a set of red intervals each of which
  is specified by its two end points.  Let $B_1,B_2,\ldots, B_m$ be a
  set of blue intervals each of which is also specified by its two end
  points. You wish to find the smallest number of blue intervals that
  {\em cover} the red intervals. A blue interval $B_j$ covers a red
  interval $R_i$ if they contain the same point $p$ on the line.  All
  intervals are close in the sense that the end points are contained
  in the interval. Describe a greedy algorithm for this problem and
  prove its correctness.


\item We saw in lecture that Borouvka's algorithm for MST can be
  implemented in $O(m \log n)$ time where $m$ is the number of edges
  and $n$ is the number of nodes. We also saw that Prim's algorithm
  can be implemented in $O(m + n \log n)$ time. Obtain an algorithm
  for MST with running time $O(m \log \log n)$ by running Borouvka's
  algorithm for some number of steps and then switching to Prim's
  algorithm. This algorithm is better than either of the algorithms
  when $m = \Theta(n)$. Formalize the algorithm, specify the
  parameters and argue carefully about the implementation and running
  time details. Briefly justify the correctness of the algorithm
  assuming that the edge weights are unique.
\end{itemize}

\vspace{1cm}

\begin{enumerate}
%\parindent 1.5em \itemsep 3ex plus 0.5fil

%----------------------------------------------------------------------
%\def\arraystretch{1.2}

%----------------------------------------------------------------------
\item Red street in the city Shampoo-Banana can be modeled as a
  straight line starting at $0$.  The street has $n$ houses at
  locations $x_1,x_2,\ldots,x_n$ on the line. The local cable company
  wants to install some new fiber optic equipment at several locations
  such that every house is within distance $r$ from one of the
  equipment locations. The city has granted permits to install the
  equipment, but only at some $m$ locations on the street given y
  locations $y_1,y_2, \ldots,y_m$. For simplicity assume that all the
  $x$ and $y$ values are distinct. You can also assume that
  $x_1 < x_2 < \ldots < x_n$ and that $y_1 < y_2 < \ldots < y_m$.
  \begin{itemize}
  \item Describe a greedy algorithm that finds the minimum number of
    equipment locations that the cable company can build to satisfy
    the desired constraint that every house is within distance $r$
    from one of them.  Your algorithm has to detect if a feasible
    solution does not exist. Prove the correctness of the
    algorithm. One way to do this by arguing that there is an optimum
    solution that agrees with the first choice of your greedy
    algorithm.
  \item {\bf Not to submit:} The cable company has realized
    subsequently that not all locations are equal in terms of the cost
    of installing equipment. Asssume that $c_j$ is the cost at
    location $y_j$.  Describe a dynamic programming algorithm that
    minimizes the total cost of installing equipment under the same
    constraint as before. Do you see why a greedy algorithm may not
    work for this cost version?
  \end{itemize}

%----------------------------------------------------------------------

\item Let $G=(V,E)$ be an edge-weighted undirected graph.  We are
  interested in computing a minimum spanning tree $T$ of $G$ to find a
  cheapest subgraph that ensures connectivity.  However, some of the
  nodes in $G$ are unrealiable and may fail. If a node fails it can disconnect
  the tree $T$ unless it is a leaf. Thus, you want to find a cheapest 
  spanning tree in $G$ in which all the unreliable nodes (which is
  a given subset $U \subset V$) are leaves. Describe an efficient 
  for this problem. Note that your algorithm should also check wither
  a feasible spanning tree satisfying the given constraint exists in $G$.

%----------------------------------------------------------------------
    \item Consider the language $L_{\text{OH}} = \{ \langle M\rangle
      \mid M \text{~halts on at least one input string}\}$. Note that
      for $\langle M\rangle \in L_{\text{OH}}$, it is not necessary
      for $M$ to {\em accept} any string; it is sufficient for it to
      {\em halt} on (and possibly rejects) some string. Prove that
      $L_{\text{OH}}$ is undecidable.

\end{enumerate}


\end{document}
