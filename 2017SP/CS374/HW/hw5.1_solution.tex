% ---------
%  Compile with "pdflatex hw0".
% --------
%!TEX TS-program = pdflatex
%!TEX encoding = UTF-8 Unicode

\documentclass[11pt]{article}
\usepackage{jeffe,handout,graphicx}
\usepackage[utf8]{inputenc}		% Allow some non-ASCII Unicode in source

%  Redefine suits
\usepackage{pifont}
\def\Spade{\text{\ding{171}}}
\def\Heart{\text{\textcolor{Red}{\ding{170}}}}
\def\Diamond{\text{\textcolor{Red}{\ding{169}}}}
\def\Club{\text{\ding{168}}}

\def\Cdot{\mathbin{\text{\normalfont \textbullet}}}
\def\Sym#1{\textbf{\texttt{\color{BrickRed}#1}}}
\newcommand{\comp}[1]{#1_{\text{comp}}}

\newtheorem{Lemma}{lemma}
\newcommand{\IsSinL}{\text{IsStringInL}}

\usepackage{tikz}
\usetikzlibrary{automata, positioning}
\usepackage{listings}

% =====================================================
%   Define common stuff for solution headers
% =====================================================
\Class{CS/ECE 374}
\Semester{Spring 2017}
\Authors{3}
\AuthorOne{Lanxiao Bai}{lbai5}
\AuthorTwo{Renheng Ruan}{rruan2}
\AuthorThree{Ho Yin Au}{hoyinau2}
%\Section{}

% =====================================================
\begin{document}

% ---------------------------------------------------------


\HomeworkHeader{5}{1}	% homework number, problem number

\begin{quote}
	Let $w \in \Sigma^*$ be a string. We say that
  $u_1,u_2,\ldots,u_h$ where each $u_i \in \Sigma^*$ is a valid split
  of $w$ iff $w = u_1u_2\ldots u_h$ (the concatenation of
  $u_1,u_2,\ldots,u_h$). Given a valid split $u_1,u_2,\ldots,u_h$ of
  $w$ we define its $\ell_3$ measure as $\sum_{i=1}^h |u_i|^3$. 

  Given a language $L \subseteq \Sigma^*$ a string $w \in L^*$ iff
  there is a valid split $u_1,u_2,\ldots,u_h$ of $w$ such that each
  $u_i \in L$; we call such a split an $L$-valid split of $w$.  Assume
  you have access to a subroutine $\IsSinL(x)$ which outputs whether
  the input string $x$ is in $L$ or not. To evaluate the running time
  of your solution you can assume that each call to $\IsSinL()$ takes
  constant time.

  Describe an efficient algorithm that given a string $w$ and access to
  a language $L$ via $\IsSinL(x)$ outputs an $L$-valid split of $w$ 
  with minimum $\ell_3$ measure if one exists.
\end{quote}
\hrule

\begin{solution}
	First of all, to reach the possible minimal measure $\ell_3$, we want to know in what occasion it is the least. And here we claim that 
	
	\textbf{Lemma 1:} If there's a set of positive integers $A = \{a_n : n \mathbb{N}\}$
		\[\sum_{i = 1}^n a^3_i \leq (\sum_{i = 1}^n a_i)^3\]
		for all $n \in \mathbb{N}$.
		
	\textbf{Proof:} We can prove this lemma by mathematical induction.
	
	Base case, when $n = 1$, 
		\[\sum_{i = 1}^1 a^3_i = a_1^3 = (\sum_{i = 1}^1 a_i)^3\]
		
	Inductive Hypothesis, suppose for all $n \leq k$,
		\[\sum_{i = 1}^n a^3_i \leq (\sum_{i = 1}^n a_i)^3\]
		
	Then when $n = k + 1$, 
		\begin{align}
			&\sum_{i = 1}^{k + 1} a^3_i = \sum_{i = 1}^k a_i^3 + a_{k + 1}\nonumber\\
			&\phantom{\sum_{i = 1}^{k + 1} a^3_i} = \sum_{i = 1}^k a_i^3 + \sum_{i = 1}^1 a_{k + 1}\nonumber\\
			&\phantom{\sum_{i = 1}^{k + 1} a^3_i} \leq \sum_{i = 1}^k a_i^3 + \sum_{i = 1}^1 a_{k + 1}^3\nonumber\\
			&\phantom{\sum_{i = 1}^{k + 1} a^3_i} \leq (\sum_{i = 1}^{k} a_i)^3 + (\sum_{i = 1}^1 a_{k + 1})^3\nonumber\\
			&\phantom{\sum_{i = 1}^{k + 1} a^3_i} < (\sum_{i = 1}^{k} a_i + a_{k + 1})^3 = (\sum_{i = 1}^{k + 1} a_i)^3\nonumber
		\end{align}
	
	Hence, we conclude that \[\sum_{i = 1}^n a^3_i \leq (\sum_{i = 1}^n a_i)^3\]
		for all $n \in \mathbb{N}$.
		
	Then, this lemma means that to reach the minimum $\ell_3$ measure, we want to have each segment of string is the shortest valid string.
	
	As a result, we can recursively split string into two valid segments in $L$ until they cannot be split any more, of there are more than $1$ way of splitting the string, $\ell_3$ need to be compared to determine the optimum solution.
	
	Hence, we define the recurrence function that returns the split of string with minimum $\ell_3$:
	\[\textsc{MinimumL3Split}(\textit{i, j}) = 
	\begin{cases}
		 \textsc{null} & \text{if $j > n$}\\
		 w[i\, ..\, j] + \textsc{MinimumL3Split}(\textit{j+1, j+1}) & \text{if $j \leq n$ and $\IsSinL(w[i\, ..\, j])$}\\
		 \textsc{MinimumL3Split}(\textit{$i, j + 1$}) & \text{if $j \leq n$ and not $\IsSinL(w[i\, ..\, j])$}
	\end{cases}	\]
	
	So with the function, we need to compute $\textsc{MinimumL3}(1, 1)$ if $|w| = n$.
	
	We can memoize all function values in a three-dimensional array $\textsc{MinimumL3}[1\, ..\,n][1\,..\,n]$. Then we see that each entry $\textsc{MinimumL3}[i, j]$ depends on its up $\textsc{MinimumL3}[i, j + 1]$ and the entry on the diagonal of its row. Thus, we can fill the array from left to right, top to bottom. The recurrence gives us the following pseudocode:
	
		\begin{center}
			\begin{algorithm}
				\textul{$\textsc{Split}(w[1\,..\,n]):$}\+
				\\	\textsc{MinimumL3}[n,n] = \textsc{Null}
				\\	for $i \gets n$ to $1$\+
				\\		for $j \gets n$ to $i$\+
				\\			if \IsSinL(\textit{w}[$i,j$])\+
				\\				 \textsc{MinimumL3Split}[$i,j$] $\gets$ w[i\, ..\, j] + \textsc{MinimumL3Split}(\textit{$j+1, j+1$})\-
				\\			else\+
				\\				\textsc{MinimumL3Split}[$i,j$] $\gets$ \textsc{MinimumL3Split}(\textit{$i, j + 1$})\-\-\-
				\\ return \textsc{MinimumL3Split}[$1,1$]
			\end{algorithm}
		\end{center}
	
	So, obvious the overall runtime of this algorithm is $O(n^2)$.
	
\end{solution}
\clearpage

\end{document}
