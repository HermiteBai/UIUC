% ---------
%  Compile with "pdflatex hw0".
% --------
%!TEX TS-program = pdflatex
%!TEX encoding = UTF-8 Unicode

\documentclass[11pt]{article}
\usepackage{jeffe,handout,graphicx}
\usepackage[utf8]{inputenc}		% Allow some non-ASCII Unicode in source

%  Redefine suits
\usepackage{pifont}
\def\Spade{\text{\ding{171}}}
\def\Heart{\text{\textcolor{Red}{\ding{170}}}}
\def\Diamond{\text{\textcolor{Red}{\ding{169}}}}
\def\Club{\text{\ding{168}}}

\def\Cdot{\mathbin{\text{\normalfont \textbullet}}}
\def\Sym#1{\textbf{\texttt{\color{BrickRed}#1}}}

\newcommand{\comp}[1]{#1_{\text{comp}}}
\newtheorem{claim}{Claim}
\def\To{\leadsto}
\def\Tostar{\mathrel{\To\!\!\!^*}}



% =====================================================
%   Define common stuff for solution headers
% =====================================================
\Class{CS/ECE 374}
\Semester{Spring 2017}
\Authors{3}
\AuthorOne{Renheng Ruan}{rruan2}
\AuthorTwo{Lanxiao Bai}{lbai5}
\AuthorThree{Ho Yin Au}{hoyinau2}

%\Section{}

% =====================================================
\begin{document}

% ---------------------------------------------------------


\HomeworkHeader{7}{3}	% homework number, problem number

\begin{quote}

\begin{enumerate}
	Given an undirected connected graph $G=(V,E)$ an edge $(u,v)$ is
called a cut edge or a bridge if removing it from $G$ results in
two connected components (which means that $u$ is in one component
and $v$ in the other). The goal in this problem is to design an efficient
algorithm to find {\em all} the cut-edges of a graph.
  \begin{itemize}
	\item What are the cut-edges in the graph shown in the figure?	
	\item Given $G$ and edge $e=(u,v)$ describe a linear-time algorithm
	that checks whether $e$ is a cut-edge or not. What is the running time
	to find all cut-edges by trying your algorithm for each edge? No proofs
	necessary for this part.
	\item Consider {\em any} spanning tree $T$ for $G$. Prove that every
	cut-edge must belong to $T$. Conclude that there can be at most $(n-1)$
	cut-edges in a given graph. How does this information improve the 
	algorithm to find all cut-edges from the one in the previous step?
	\item Suppose $T$ is a spanning tree of $G$ rooted at $r$.
	Prove that an edge $(u,v)$ in $T$ where $u$ is
	the parent of $v$ is a cut-edge iff there is no edge in $G$, other
	than $(u,v)$, 
	with one end point in $T_v$ (sub-tree of $T$ rooted at $v$)
	and one end point outside $T_v$.
	\item Use the property in the preceding part to design a linear-time
	algorithm that outputs all the cut-edges of $G$. You don't have to 
	prove the correctness of the algorithm but you should point out how
	your algorithm ensures the desired property. {\em Hint:} Consider
	a DFS tree $T$ and some additional information you can compute
	during DFS. You may want to run DFS on the example graph with the
	cut edges identified.
\end{itemize}

\end{enumerate}

\end{quote}
\hrule

\begin{solution}
	\mbox{}
	\begin{enumerate}[(a)]
		\item $(e,g)$, $(f,j)$, and $(h,l)$ are the cut-edges.
		\item By removing e from G and use whatever-first search to determine whether u is still connected to v. If u still connected to v, then e is not a cut-edge, and e is a cut-edge if u not connected to v. Whatever-first search takes $O(m)$ time, and take $O(m^2)$ if we go though each of the m edges.
		\item For any edge e $\notin$ T, removing e would not makes the graph to be disconnected because there is always a path in the spanning tree that connects any two vertices in $G$.\\
		There can be at most $n - 1$ cut-edges in a given graph because only the edges in the spanning tree can be cut-edges, and the spanning tree has n-1 edges.\\
		Since the number of candidate cut-edges drop from $m$ to $n-1$, then the algorithm takes $O(mn)$ time by going though each of the $n-1$ edges.
		\item Proving by Contrapositive: We assume that there is an edge $e$ in G with one end point in $T_v$ (sub-tree of $T$ rooted at $v$) and one end point outside $T_v$. We know that T is connected, then removing $(u,v)$ from $T$ leaves it in two connected components: one contains u and another contains v ($T_v$). The graph will be $G-(u,v)$ which is a single connected component in G by the edge $e$ connecting those two components. Hence, $(u,v)$ is not a cut-edge.\\
		By removing the edge $(u,v)$ from $T$. We have two connected components: one contains u and another contains v ($T_v$). $T_v$ is a connected component of $G-(u,v)$ because $T_v$ does not have any edge leaving it in $G-(u,v)$. And u is also a connected component which is not connect $T_v$ in $G-(u,v)$. Hence, $(u,v)$ is a cut-edge.
		\item Let $T$ be a DFS tree.\\
		We define each node u:
		\begin{center}
			\[
			low(u) = min \begin{cases}
			$pre(u)$\\
			$pre(w)$ & \text{$(v,w)$is a back edge for $v \notin T_u$}\\
			\end{cases}
			\]
		\end{center}
		The following procedure records the previsit time $pre(u)$ and predecessor $pred(u)$, and compute $low(u)$ for each $u$ in DFS.
		\begin{center}
			\begin{algorithm}			
				\textul{$\textsc{LowDFS}(\textit{G})$:}\+
				\\ for all $u \in V$ \+
				\\  	$pred(u)$ $\gets$ NULL\-
				\\ chose an arbitrary vertex $u \in V$
				\\ LowDFSwizU(G,u)
			\end{algorithm}
		\end{center}
		\begin{center}
			\begin{algorithm}			
				\textul{$\textsc{LowDFSwizU}(\textit{G,u})$:}\+
				\\ change u to a visited vertex
				\\ $time \gets time+1$
				\\ $pre(u) \gets time$
				\\ $low(u) \gets pre(u)$
				\\ for each edge $(u,v)$ in $Adj(u)$\+
				\\ if $v$ is not marked as visited\+
				\\ 		LowDFSwizU(G,v)
				\\ 		$low(u) \gets min\{low(u),low(v)\}$
				\\ 		$pred(v) \gets u$\-
				\\ else\+
				\\ 		$low(u) \gets min\{low(u),pre(v)\}$\-
			\end{algorithm}
		\end{center}
		The procedure above is DFS with predecessor and previsit time, and with constant time at every step to calculate $low(u)$. So the running time is same as DFS which is $O(m+n)$.\\
		And the AllCuts function returns the set of all of the cut-edges:
		\begin{center}
			\begin{algorithm}			
				\textul{$\textsc{AllCuts}(\textit{G})$:}\+
				\\ $pre(),low(),pred() \gets$ LowDFS(G)
				\\ $S \gets \emptyset$
				\\ for each vertex  $v \in V$ \+
				\\ 		if $low(v) = pre(v)$ and $pred(v) \neq $ NULL\+
				\\ 			add $(pred(v),v)$ to $S$\-\-
				\\ 		return $S$\-
			\end{algorithm}
		\end{center}
	\end{enumerate}
	The algorithm we have takes $O(m+n)$ time, and $Allcuts(G)$ takes $O(n)$ time. Thus, the total running time is $O(m+n)$.\\
	\\
	Proving by induction to show how $low(u)$ is correctly computed:\\
	\\
	Base Case:\\
	u has no children, then $low(u) = pre(u)$, which is correct. Or there are back edges from u, which calculate the min $\{pre(w):(u,w) is a back edge\}$.\\
	\\
	Inductive hypothesis:\\
	Assume $low(u)$ is calculated correctly for all children v of u.\\
	\\
	Inductive Step:\\
	If there are no back edges leaving $T_u$ then $low(u) = pre(u)$. If $low(u)$ is achieved by $pre(w)$ where $(u,w)$ is a back edge, then the "else" clause will calculate $low(u)$ correctly. If $low(u)$ is achieved by $pre(w)$ where $(v,w)$ is a back edge for $v \in T_u$, $v \notin u$, then the line after the recursive call LowDFSwizU(G,v) will calculate $low(u)$ correctly by the inductive hypothesis.\\
	\\
The solution template is taken from\\ https://www.coursehero.com/file/13235458/hw7-solutionspdf/.\\
\end{solution}

\end{document}
