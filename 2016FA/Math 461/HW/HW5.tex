\documentclass[11pt]{article}
%-----------Packeges---------------%
\usepackage{amsmath}
\usepackage{amssymb}
\usepackage{amsfonts}
\usepackage{tocloft}
\usepackage{float}
\usepackage{graphicx}
\usepackage[bookmarks=true]{hyperref}
\usepackage{fancyhdr}


%----------Definition & Theorem----%
\newtheorem{definition}{Definition}[subsection]
\newtheorem{theorem}{Theorem}[subsection]
\newtheorem{proposition}{Proposition}[subsection]
\newtheorem{lemma}{Lemma}[subsection]
\newtheorem{corollary}{Corollary}[subsection]

\pagestyle{fancy}
\fancyhead[L]{Math 461}
\fancyhead[C]{HW5}
\fancyhead[R]{Lanxiao Bai(lbai5)}
\begin{document}
	\paragraph{4.1}\textbf{Solution:}
	It is possible that $X = -2, -1, 0, 1, 2, 4$.
		\begin{align}
			&P(-2) = \frac{8}{14}\frac{7}{13} = 4/13\nonumber\\
			&P(-1) = \frac{8}{14}\frac{2}{13} + \frac{2}{14}\frac{8}{13} = 16/91\nonumber\\
			&P(0) = \frac{2}{14}\frac{1}{13} = 1/91\nonumber\\
			&P(1) = \frac{4}{14}\frac{8}{13} + \frac{8}{14}\frac{4}{13} = 32/91\nonumber\\
			&P(2) = \frac{4}{14}\frac{2}{13} + \frac{2}{14}\frac{4}{13} = 8/91\nonumber\\
			&P(4) = \frac{4}{14}\frac{3}{13} = 6/91\nonumber
		\end{align}
	\paragraph{4.4}\textbf{Solution:} Since there are 5 men and 5 women, $\max(X) = 6$.
		\begin{align}
			&P\{X=1\} = 5\frac{9!}{10!} = 1/2\nonumber\\
			&P\{X=2\} = 5\cdot 5\frac{8!}{10!} = 5/18\nonumber\\
			&P\{X=3\} = 5\cdot 4\cdot 5\frac{7!}{10!} = 5/36\nonumber\\
			&P\{X=4\} = 5\cdot 4\cdot 3\cdot 5\frac{6!}{10!} = 5/84\nonumber\\
			&P\{X=5\} = 5\cdot 4\cdot 3\cdot 2\cdot 5\frac{5!}{10!} = 5/252\nonumber\\
			&P\{X=6\} = 5\cdot 4\cdot 3\cdot 2\cdot 1\cdot 5\frac{4!}{10!} = 1/252\nonumber
		\end{align}
	\paragraph{4.5}
		\subparagraph{Solution:}
			If n is even, we have possibilities of $(0, n), (1, n - 1), (2, n - 2), \cdots, (n/2, n/2), \cdots, (n - 2, 2), (n - 1, 1), (n, 0)$.
			
			If n is odd we have possibilities of $(0, n), (1, n - 1), (2, n - 2), \cdots, (m - 1, m), (m , m - 1), \cdots, (n - 2, 2), (n - 1, 1), (n, 0)$.
			
			Thus, $X \in \{0, 1, 2, 3, ... , n\}$.
	\paragraph{4.13}
		\subparagraph{Solution:}
		\begin{align}
			&P\{X = 0\} = 0.7\cdot 0.4 = 0.28\nonumber\\
			&P\{X = 500\} = 0.3\cdot 0.5\cdot 0.4 + 0.7\cdot 0.5\cdot 0.6 = 0.27\nonumber\\
			&P\{X = 1000\} = 0.3\cdot 0.5\cdot 0.4 + 0.7\cdot 0.5\cdot 0.6 + 0.3\cdot 0.5 \cdot 0.6 \cdot 0.5 = 0.315\nonumber\\
			&P\{X = 1500\} = 0.3\cdot 0.5 \cdot 0.6 \cdot 0.5\cdot 2 = 0.09\nonumber\\
			&P\{X = 2000\} = 0.3\cdot 0.5 \cdot 0.6 \cdot 0.5 = 0.045\nonumber
		\end{align}
	\paragraph{4.14}
		\subparagraph{Solution:}
			\begin{align}
				&P\{X = 0\} = (1/5)((1/4) + (2/4) + (3/4) + (4/4)) = 1/2\nonumber\\
				&P\{X = 1\} = (1/5)((3/4)(1/3) + (2/4)(2/3) + (1/4)(3/3)) = 1/6\nonumber\\
				&P\{X = 2\} = (1/5)((3/4)(2/3)(1/2) + (2/4)(1/3)(2/2)) = 1/12\nonumber\\
				&P\{X = 3\} = (1/5)(3/4)(2/3)(1/2) = 1/20\nonumber\\
				&P\{X = 4\} = 1/5\nonumber
			\end{align}
	\paragraph{4.17}\textbf{Solution:}
		\subparagraph{(a)}
			\begin{align}
				&P\{X = 1\} = 1/2 - 1/4 = 1/4\nonumber\\
				&P\{X = 2\} = 11/12 - 3/4 = 1/6\nonumber\\
				&P{X = 3} = 1 - 11/12 = 1/12\nonumber
			\end{align}
		\subparagraph{(b)}
			\[P(1/2 < X < 3/2) = 5/8 - 1/8 = 1/2\]
	\paragraph{4.21}\textbf{Solution:}
		\subparagraph{(a)}
			\[E[X] \geq E[Y]\] because the bus with higher weight is more likely to be chosen for X. 
		\subparagraph{(b)}
			\[E[X] = \sum xp(x) = 40 \frac{40}{148} + 33 \frac{33}{148} + 25 \frac{25}{148} + 50 \frac{50}{148} = 39.28\]
			\[E[Y] = \sum yp(y) = 148 \frac{1}{4} = 37\]
	\paragraph{4.23}\textbf{Solution:}
		\subparagraph{(a)}
			Suppose the best strategy is to buy $k$ ounces at the start of the week with lower price and sell them at last with higher price. Since I have only $1000$ dollars, so $0 \leq k \leq 500$. 
			
			As a result, we can expect to have \[E[X] = \frac{1}{2}(1000 - 2k + 4k + 1000 - 2k + k) = 1000 + k/2\]
			
			So when $k = 500$, expectation reaches its maximum value $E[X]_{\mathrm{max}} = 1250$.
			
			Thus, the best strategy is to buy $500$ ounces at the beginning of the week and then sell them all out at last.
		\subparagraph{(b)}
			Similarly, we can calculate the expectation
			\[E[Y] = \frac{1}{2}(k + (1000 - 2k)/1 + k + (1000 - 2k)/4) = 625 - k/4\]
			
			So when $k = 0$, $E[Y]_{\mathrm{min}} = 625$. 
			
			So the best strategy is to buy all commodity at last.
	\paragraph{4.32}\textbf{Solution:}
	Since any person has $0.1$ probability to be ill, then
			\[P(\mathrm{has\ positive}) = 1 - (0.9)^{10}\]
			\[E[X] = (1 + 10)(0.9)^{10} + 0(1 - (0.9)^{10}) = 3.835\]
	\paragraph{4.35}\textbf{Solution:}
		\subparagraph{(a)}
			Since $P(\mathrm{same}) = 2\cdot \frac{1}{2} \cdot \frac{4}{9} = 4/9$, $P(\mathrm{Not\ same}) = 1 - 4/9 = 5/9$, \[E[X] = 1.1 \cdot \frac{4}{9} + (-1.0) \cdot \frac{5}{9} = -\frac{0.6}{9} = -\frac{1}{15}\]
		\subparagraph{(b)}
			\[\mathrm{Var}(X) = E[X^2] - (E[X])^2 = (1.1^2 \cdot \frac{4}{9} + (-1.0)^2 \cdot \frac{5}{9}) - (-1/15)^2 = 1.089\]
	\paragraph{4.37}\textbf{Solution:}
		\[\mathrm{Var}[X] = E[X^2] - (E[X])^2 = (40^2 \frac{40}{148} + 33^2 \frac{33}{148} + 25^2 \frac{25}{148} + 50^2 \frac{50}{148}) - (39.28)^2 = 82.2\]
		\[\mathrm{Var}[Y] = E[Y^2] - (E[Y])^2 = ((40^2 + 33^2 + 25^2 + 50^2)\frac{1}{4}) - (37)^2 = 84.5\]
\end{document}
