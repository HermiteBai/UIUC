\documentclass[11pt]{article}
%-----------Packeges---------------%
\usepackage{amsmath}
\usepackage{amssymb}
\usepackage{amsfonts}
\usepackage{tocloft}
\usepackage{float}
\usepackage{graphicx}
\usepackage[bookmarks=true]{hyperref}
\usepackage{fancyhdr}


%----------Definition & Theorem----%
\newtheorem{definition}{Definition}[subsection]
\newtheorem{theorem}{Theorem}[subsection]
\newtheorem{proposition}{Proposition}[subsection]
\newtheorem{lemma}{Lemma}[subsection]
\newtheorem{corollary}{Corollary}[subsection]

\pagestyle{fancy}
\fancyhead[L]{Math 461}
\fancyhead[C]{HW 4}
\fancyhead[R]{Lanxiao Bai(lbai5)}
\begin{document}
	\paragraph{3.64}
		\subparagraph{Solution:} Denote the probability of correct for strategy $a$ as $P(a)$ and $P(b)$ for the probability of correct for strategy $b$.
		
			Then \[P(a) = p\] and \[P(b) = p^2 + 2p(1 - p) \cdot 1/2 = p\]
			
						
			So we can conclude that those two strategies have the same possibility to give the correct answer.
	\paragraph{3.66} 
		\subparagraph{Solution:} 
		\[P(a) = p_1 p_2 p_5 + p_3 p_4 p_5 - p_1 p_2 p_3 p_4 p_5\]
		\[P(b) = p_1p_4 + p_1 p_3 p_5 + p_2p_5 + p_2p_3p_4 - p_1p_3p_4p_5 - p_1p_2p_4p_5 - p_1p_2p_3p_4\]
		\[ - p_1 p_2 p_3 p_5 - p_2 p_3 p_4 p_5 + 2p_1 p_2 p_3 p_4 p_5\]
	\paragraph{3.78}\textbf{Solution:}
		\subparagraph{(a)}\[P(4) = P(\mathrm{A\ win}) + P(\mathrm{B\ win}) = 2p^3(1 - p) + 2(1 - p)^3p\]
		\subparagraph{(b)}
		\begin{align}
			&P(\mathrm{A\ win}) = \sum_{i = 1}^{\infty} P(\mathrm{A\ win\ 2i})\nonumber\\
			&\phantom{P(\mathrm{A\ win})} = \sum_{i = 1}^{\infty} 2^{i - 1}p^{i + 1}(1 - p)^{i - 1} = \frac{p^2}{1 - 2p(1 - p)}
		\end{align}
		
	\paragraph{3.83}
		\subparagraph{(a)}
			\[P(\mathrm{red}) = 1/2 \cdot (2/3 + 1/3) = 1/2\]
		\subparagraph{(b)}
			\[P(\mathrm{red} | \mathrm{first\ 2\ red}) = \frac{\frac{1}{2}(1/3)^3 + \frac{1}{2}(2/3)^3}{\frac{1}{2}(1/3)^2 + \frac{1}{2}(2/3)^2} = 3/5\]
		\subparagraph{(c)}
			\[P(\mathrm{H} | \mathrm{first\ 2\ red}) = \frac{\frac{4}{9}\frac{1}{2}}{\frac{4}{9}\frac{1}{2} + \frac{1}{9}\frac{1}{2}} = 4/5\]
	\paragraph{3.84}
		\subparagraph{(a)}
			\begin{align}
			&P(A) = (\sum_{i = 1}^\infty ((2/3)^3)^{i - 1})(1/3)\nonumber\\ 
			&\phantom{P(A)} = \frac{1}{3} \sum_{i = 1}^\infty (\frac{8}{27})^{i - 1}\nonumber\\
			&\phantom{P(A)} = \frac{1}{3} \frac{27}{19} (1 - \lim_{n \rightarrow \infty} (\frac{8}{27})^n)\nonumber\\ 
			&\phantom{P(A)} = 9/19\nonumber
			\end{align}
			
			\begin{align}
				&P(B) = (\sum_{i = 1}^\infty ((2/3)^3)^{i - 1})(2/3)(1/3)\nonumber\\
				&\phantom{P(B)} = 6/19\nonumber
			\end{align}
			
			\begin{align}
				&P(C) = 	(\sum_{i = 1}^\infty ((2/3)^3)^{i - 1})(2/3)^2(1/3)\nonumber\\
				&\phantom{P(C)} = 4/19\nonumber
			\end{align}

		\subparagraph{(b)}
			\begin{align}
				&P(A) = 1/3 + (2/3)(7/11)(6/10)(4/9) + (2/3)(7/11)(6/10)(5/9)(4/8)(3/7)(4/6)\nonumber\\
				&\phantom{P(A)} = 7/15\nonumber\\
				&P(B) = (8/12)(4/11) + (2/3)(7/11)(6/10)(5/9)(4/8)\nonumber\\
				&+ (2/3)(7/11)(6/10)(5/9)(4/8)(3/7)(2/6)(4/5)\nonumber\\
				&\phantom{P(B)} = 53/165\nonumber\\
				&P(C) = (8/12)(7/11)(4/10) + (2/3)(7/11)(6/10)(5/9)(4/8)(4/7)\nonumber\\
				& + (2/3)(7/11)(6/10)(5/9)(4/8)(3/7)(2/6)(1/5)(4/4)\nonumber\\
				&\phantom{P(C)} = 7/33\nonumber
			\end{align}
	\paragraph{3.13 (Theoretical Exercises)}
		\subparagraph{Solution:}
			If the initial flip lands on head, then A will win with $P_{n - 1, m}$, if it lands on tail, then it's B's turn to flip the coin with the possibility $P_{m , n}$ with the possibility $(1 - P_{m , n})$.
			
			As a result, \[P_{n,m} = pP_{n-1, m} + (1-p)(1-P{m,n})\]
	\paragraph{4.1}\textbf{Solution:}
	It is possible that $X = -2, -1, 0, 1, 2, 4$.
		\begin{align}
			&P(-2) = \frac{8}{14}\frac{7}{13} = 4/13\nonumber\\
			&P(-1) = \frac{8}{14}\frac{2}{13} + \frac{2}{14}\frac{8}{13} = 16/91\nonumber\\
			&P(0) = \frac{2}{14}\frac{1}{13} = 1/91\nonumber\\
			&P(1) = \frac{4}{14}\frac{8}{13} + \frac{8}{14}\frac{4}{13} = 32/91\nonumber\\
			&P(2) = \frac{4}{14}\frac{2}{13} + \frac{2}{14}\frac{4}{13} = 8/91\nonumber\\
			&P(4) = \frac{4}{14}\frac{3}{13} = 6/91\nonumber
		\end{align}
	\paragraph{4.4}\textbf{Solution:} Since there are 5 men and 5 women, $\max(X) = 6$.
		\begin{align}
			&P\{X=1\} = 5\frac{9!}{10!} = 1/2\nonumber\\
			&P\{X=2\} = 5\cdot 5\frac{8!}{10!} = 5/18\nonumber\\
			&P\{X=3\} = 5\cdot 4\cdot 5\frac{7!}{10!} = 5/36\nonumber\\
			&P\{X=4\} = 5\cdot 4\cdot 3\cdot 5\frac{6!}{10!} = 5/84\nonumber\\
			&P\{X=5\} = 5\cdot 4\cdot 3\cdot 2\cdot 5\frac{5!}{10!} = 5/252\nonumber\\
			&P\{X=6\} = 5\cdot 4\cdot 3\cdot 2\cdot 1\cdot 5\frac{4!}{10!} = 1/252\nonumber
		\end{align}
	\paragraph{4.5}
		\subparagraph{Solution:}
			If n is even, we have possibilities of $(0, n), (1, n - 1), (2, n - 2), \cdots, (n/2, n/2), \cdots, (n - 2, 2), (n - 1, 1), (n, 0)$.
			
			If n is odd we have possibilities of $(0, n), (1, n - 1), (2, n - 2), \cdots, (m - 1, m), (m , m - 1), \cdots, (n - 2, 2), (n - 1, 1), (n, 0)$.
			
			Thus, $X \in \{0, 1, 2, 3, ... , n\}$.
	\paragraph{4.13}
		\begin{align}
			&P\{X = 0\} = 0.7\cdot 0.4 = 0.28\nonumber\\
			&P\{X = 500\} = 0.3\cdot 0.5\cdot 0.4 + 0.7\cdot 0.5\cdot 0.6 = 0.27\nonumber\\
			&P\{X = 1000\} = 0.3\cdot 0.5\cdot 0.4 + 0.7\cdot 0.5\cdot 0.6 + 0.3\cdot 0.5 \cdot 0.6 \cdot 0.5 = 0.315\nonumber\\
			&P\{X = 1500\} = 0.3\cdot 0.5 \cdot 0.6 \cdot 0.5\cdot 2 = 0.09\nonumber\\
			&P\{X = 2000\} = 0.3\cdot 0.5 \cdot 0.6 \cdot 0.5 = 0.045\nonumber
		\end{align}
	\paragraph{4.14}
		\paragraph{Solution:}
			\begin{align}
				&P\{X = 0\} = (1/5)((1/4) + (2/4) + (3/4) + (4/4)) = 1/2\nonumber\\
				&P\{X = 1\} = (1/5)((3/4)(1/3) + (2/4)(2/3) + (1/4)(3/3)) = 1/6\nonumber\\
				&P\{X = 2\} = (1/5)((3/4)(2/3)(1/2) + (2/4)(1/3)(2/2)) = 1/12\nonumber\\
				&P\{X = 3\} = (1/5)(3/4)(2/3)(1/2) = 1/20\nonumber\\
				&P\{X = 4\} = 1/5\nonumber
			\end{align}
	\paragraph{4.17}\textbf{Solution:}
		\subparagraph{(a)}
			\begin{align}
				&P\{X = 1\} = 1/2 - 1/4 = 1/4\nonumber\\
				&P\{X = 2\} = 11/12 - 3/4 = 1/6\nonumber\\
				&P{X = 3} = 1 - 11/12 = 1/12\nonumber
			\end{align}
		\subparagraph{(b)}
			\[P(1/2 < X < 3/2) = 5/8 - 1/8 = 1/2\]
\end{document}
