\documentclass[11pt]{article}
%-----------Packeges---------------%
\usepackage{amsmath}
\usepackage{amssymb}
\usepackage{amsfonts}
\usepackage{tocloft}
\usepackage{float}
\usepackage{graphicx}
\usepackage[bookmarks=true]{hyperref}
\usepackage{fancyhdr}


%----------Definition & Theorem----%
\newtheorem{definition}{Definition}[subsection]
\newtheorem{theorem}{Theorem}[subsection]
\newtheorem{proposition}{Proposition}[subsection]
\newtheorem{lemma}{Lemma}[subsection]
\newtheorem{corollary}{Corollary}[subsection]

\pagestyle{fancy}
\fancyhead[L]{Math 417}
\fancyhead[C]{HW10}
\fancyhead[R]{Lanxiao Bai(lbai5)}
\begin{document}
	\paragraph{5.1.4}
		\subparagraph{Claim:}Let $G$ be any group and $H$ any subgroup. Then $G$ acts on the set $G/H$ of left cosets of $H$ in $G$ by left multiplication. Then action is transitive.
		\subparagraph{Proof:} $\forall aH \in G/H, \forall g \in G, g\cdot G/H = gaH = (ga)H$, since $ga \in G$ by closure, $gaH \in G/H$. $\forall bH \in G/H, g = ba^{-1} \in G$ so that $gaH = bH$, so we proved that the number of orbit $\mathcal{O}(x)$ is $1$.
		
		As a result, $G$ acts on the set $G/H$ of left cosets of $H$ in $G$ by left multiplication is transitive.$\blacksquare$
	\paragraph{5.1.6}
		\subparagraph{Claim:}Let $G$ act on $X$, and let $x \in X$. We have $\mathrm{Stab}(x) \subseteq G$ and if x and y are in the same orbit, then the subgroups $\mathrm{Stab}(x)$ and $\mathrm{Stab}(y)$ are conjugate subgroups.
		\subparagraph{Proof:}Take $s \in \mathrm{Stab}(x) = \{g \in G: g\cdot x = x\}$, obviously $s \in G $, so $ \mathrm{Stab}(x) \subseteq G$.
		
			If $x$ and $y$ are in the same orbit, then $\exists g \in G$ that $y = g\cdot x = gxg^{-1}$. Then for $\mathrm{Stab}(x) = \{g \in G: g\cdot x = x\} = \{g \in G: gxg^{-1} = x\}$ and $\mathrm{Stab}(y) = \{g \in G: g\cdot y = y\} = \{g \in G: gyg^{-1} = y\}$. $\forall g' \in G, g'\mathrm{Stab}(x)g'^{-1} = \{g \in G: g'g\cdot xg^{-1}g'^{-1} = x\} = \{g \in G: g'gxg^{-1}g'^{-1} = x\} = \{g \in G: g'gxg^{-1}g'^{-1} = x\} = \{g \in G: g\cdot y = y\} = \mathrm{Stab}(y)$.$\blacksquare$
	\paragraph{5.2.2}
		\subparagraph{Claim:} $2$ red beads, $2$ green beads and $2$ violet heads can make $11$ necklaces.
		\subparagraph{Proof:} Since the symmetries consist $6$ rotational symmetries and $6$ reflectional symmetries. We choose $G = D_6$, so $|G| = 12$. And since \[|X| = \frac{6!}{2!2!2!} = 90,\] so $|\mathrm{Fix}(e)| = 90, |\mathrm{Fix}(r^3)| = 6, |\mathrm{Fix}(a)| = |\mathrm{b}| = |\mathrm{Fix}(c)| = |\mathrm{Fix}(d)| = |\mathrm{Fix}(e)| = |\mathrm{Fix}(f)| = 6$.
		
		Thus, by Burnside's Lemma, \[n = \frac{1}{12}(90 + 7 \cdot 6) = 11.\blacksquare\]	
		\paragraph{5.3.7}
			\subparagraph{Claim:} $\mathrm{Aut}(\mathbb{Z}_2 \times \mathbb{Z}_2) \cong S_3.$			
			\subparagraph{Proof 1:} Since $\mathbb{Z}_2 \times \mathbb{Z}_2$ can be grouped into a vector of $2D$ vector, which is basically a $2 \times 2$ matrix, in which each elements can only be either $1$ or $0$. So, obviously, the only way to transform a $2 \times 2$ matrix to a $2 \times 2$ matrix is to multiply a $2 \times 2$ matrix, which means $\mathrm{Aut}(\mathbb{Z}_2 \times \mathbb{Z}_2) \subseteq \mathrm{GL}(2, \mathbb{R})$. Of them, only $6$ matrices are invertible including:
					\[
						\begin{bmatrix}
							1 & 0\\
							0 & 1
						\end{bmatrix}
						,
						\begin{bmatrix}
							1 & 0\\
							1 & 1
						\end{bmatrix}
						,
						\begin{bmatrix}
							1 & 1\\
							0 & 1
						\end{bmatrix}
						,
						\begin{bmatrix}
							1 & 1\\
							1 & 0
						\end{bmatrix}
						,
						\begin{bmatrix}
							0 & 1\\
							1 & 0
						\end{bmatrix}
						,
						\begin{bmatrix}
							0 & 1\\
							1 & 1
						\end{bmatrix}
					\]
					Those matrices can be easily checked to be a group by doing the multiplication.
					
					Since there're only $2$ kinds of groups of order $6$ up to isomorphism $S_3$ and $\mathbb{Z}_6$. Since we can check that 
						\[
							\begin{bmatrix}
								1 & 1\\
								1 & 0
							\end{bmatrix}
							\cdot
							\begin{bmatrix}
								1 & 1\\
								0 & 1
							\end{bmatrix}
							\neq
							\begin{bmatrix}
								1 & 1\\
								0 & 1
							\end{bmatrix}
							\cdot
							\begin{bmatrix}
								1 & 1\\
								1 & 0
							\end{bmatrix}
						\]
						So we know $\mathrm{Aut}(\mathbb{Z}_2 \times \mathbb{Z}_2)$ is not commutative, which means it isn't isomorphic to $\mathbb{Z}_6$, and thus we know that $\mathrm{Aut}(\mathbb{Z}_2 \times \mathbb{Z}_2) \cong S_3.\blacksquare$
			\subparagraph{Proof 2:} Since in any group mapping, we always send identity to identity in order to preserve the structure. So what really matters is that to which we send $\{a, b, c\}$ to from the group of $\mathbb{Z}_2 \times \mathbb{Z}_2 = \{e, a, b, c\}$. Then for each homomorphism there exists a unique permutation in $S_3$ to fulfill this requirement.
			
			Then $\forall g \in S_3$, by permutation composition, $g$ acts on $\{a, b, c\}$ sends bijectively send $a, b, c$ to $\{a, b, c\}$ so we can see such permutation determines an automorphism by definition.
			
			As a result, $\mathrm{Aut}(\mathbb{Z}_2 \times \mathbb{Z}_2) \cong S_3.\blacksquare$
\end{document}
