\documentclass{article}[11pt]
%-----------Packeges---------------%
\usepackage{amsmath}
\usepackage{amssymb}
\usepackage{amsfonts}
\usepackage{tocloft}
\usepackage{float}
\usepackage{graphicx}
\usepackage[bookmarks=true]{hyperref}
\usepackage{fancyhdr}


%----------Definition & Theorem----%
\newtheorem{definition}{Definition}[subsection]
\newtheorem{theorem}{Theorem}[subsection]
\newtheorem{proposition}{Proposition}[subsection]
\newtheorem{lemma}{Lemma}[subsection]
\newtheorem{corollary}{Corollary}[subsection]

\pagestyle{fancy}
\fancyhead[L]{Math 417}
\fancyhead[C]{HW3}
\fancyhead[R]{Lanxiao Bai(lbai5)}
\begin{document}
	\paragraph{1.6.9}\textbf{Solution: }
		\subparagraph{Claim:} If prime number $p \mid a_1a_2\cdots a_r$, then p divides one of the factors.
		\subparagraph{Proof:} Since $p \mid a_1a_2\cdots a_r$, we know that $a_1a_2\cdots a_r = q_0p$, $q \in \mathbb{Z}$. And without losing generality we can denote the factor as $a_i$, thus $a_i = qp$.
		
		For base case, if $p \mid a_1$, $p \mid a_1$ is obviously true.
		
		Suppose when $r = k$, namely, $p \mid a_1a_2\cdots a_k$, p divides one of the factors is true, then when $r = k + 1$, if $p \mid a_1a_2\cdots a_{k + 1}$, so $a_1a_2\cdots a_{k + 1} = q_1p$. Then there're 2 cases.
		
		If $p \mid a_{k+1}$, the statement is proved.
		
		If $p \mid a_1a_2\cdots a_k$, the hypothesis gives that $p \mid a_i$.
		
		So when $r = k+1$, namely, $p \mid a_1a_2\cdots a_{k+1}$, p divides one of the factors is true.
		
		Thus, we can conclude that if prime number $p \mid a_1a_2\cdots a_r$, then p divides one of the factors.$\blacksquare$
	\paragraph{1.7.4}\textbf{Solution: }
		\subparagraph{Claim:} $[4^{237}] = [4]$.
		\subparagraph{Proof:} In $\mathbb{Z}_5$, $[4^2] = [1]$. Thus $[4^{2n}] = [1]$, so $[4^{236}] = [1]$. And since $[4] = [-1]$, $[4^{237}] = [4^{236}][4] = [1][-1] = [-1] = [4].\blacksquare$
	\paragraph{1.7.11}\textbf{Solution: }
		\subparagraph{Claim:} If $a$ is relatively prime to $n$ and there are integers $s$ and $t$ so that $as + nt = 1$. The inverse of $[a]$ is $[s]$.
		\subparagraph{Proof:} Since $a$ is relatively prime to $n$ and $as + nt = 1$, we have $as - 1 = nt$. So $n \mid (as - 1)$ and as a result $as \equiv 1 \mod n$. Which means that $[as] = [1]$, so $[a][s] = [1]$.
		
		Thus, $[s]$ is the inverse of $[a]$ in $\mathbb{Z}_n\blacksquare$.
	\paragraph{1.7.14}\textbf{Solution: }
		\subparagraph{(a)} 
			\textbf{Claim: }$\forall b \in \mathbb{Z}$, $ax\equiv b\mod n$ has a solution.\\
			
			\textbf{Proof: } For all integer a, b, if we want to make $ax \equiv b \mod n$ holds, $ax - b = qn, q \in \mathbb{Z} \Rightarrow ax - qn = b$ must hold. Therefore, $g.c.d(a, n) \mid b$ must be true. Since $a$ and $n$ are relatively prime, $g.c.d(a, n) = 1$, the statement above must be true.
		
		So we can conclude that $\forall b \in \mathbb{Z}$, $ax\equiv b\mod n$ has a solution$\blacksquare$.
		\subparagraph{(b)}
			Base on the logical deduction above, we need to find a pair of integer $(s, r)$ with the inverse of Euclidean Algorithm so that $sa + rn = 1$, so $(bs)a + (br)n = b$, and one $x_0 = bs$. And all the solutions become a set $\{x | x = kn + x_0, k \in \mathbb{Z}\}$.
		\subparagraph{(c)}
			\textbf{Claim:} For $8x \equiv 12 \mod 125$, $x = 64$.\\
			
			\textbf{Proof:} Since $8x \equiv 12 \mod 125$, $8x - 12 = 125q, q \in \mathbb{Z} \Rightarrow 8x - 125q = 12$
			
			So apply Euclidean Algorithm to 8 and 125 first:
			\begin{align}
				&125 = 8 \times 15 + 5\nonumber\\
				&8 = 5 \times 1 + 3\nonumber\\
				&5 = 3 \times 1 + 2\nonumber\\
				&3 = 2 \times 1 + 1\nonumber\\
				&2 = 1 \times 2\nonumber
			\end{align}
				
			Thus,
			\begin{align}
				&\phantom{\Rightarrow\ \ }125 - 8 \times 15 = 8 - 3\nonumber\\
				&\Rightarrow125 - 8 \times 16 = -3\nonumber\\
				&\Rightarrow8 \times 16 - 125 = 3\nonumber\\
				&\Rightarrow8 \times 16 - 125 = 5 - 2\nonumber\\
				&\Rightarrow8 \times 16 - 125 = (125 - 8 \times 15) - 2\nonumber\\
				&\Rightarrow8 \times 31 - 125 \times 2 = -2\nonumber\\
				&\Rightarrow125 \times 2 - 8 \times 31 = 3 - 1 = (8 - 5) - 1\nonumber\\
				&\Rightarrow125 \times 2 - 8 \times 32 = - 5 - 1 = -6\nonumber\\
				&\Rightarrow125 \times 4 - 8 \times 64 = -12\nonumber\\
				&\Rightarrow8 \times 64 - 125 \times 4 = 12\nonumber
			\end{align} 
			As a result, $x_0 = 64$. And all solutions consist a set $\{x | x = 64 + 125k, k \in \mathbb{Z}\}\blacksquare$.
\end{document}