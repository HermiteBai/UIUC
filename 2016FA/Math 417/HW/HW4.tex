\documentclass[11pt]{article}
%-----------Packeges---------------%
\usepackage{amsmath}
\usepackage{amssymb}
\usepackage{amsfonts}
\usepackage{tocloft}
\usepackage{float}
\usepackage{graphicx}
\usepackage[bookmarks=true]{hyperref}
\usepackage{fancyhdr}


%----------Definition & Theorem----%
\newtheorem{definition}{Definition}[subsection]
\newtheorem{theorem}{Theorem}[subsection]
\newtheorem{proposition}{Proposition}[subsection]
\newtheorem{lemma}{Lemma}[subsection]
\newtheorem{corollary}{Corollary}[subsection]

\pagestyle{fancy}
\fancyhead[L]{Math 417}
\fancyhead[C]{HW4}
\fancyhead[R]{Lanxiao Bai(lbai5)}

\begin{document}
	\paragraph{1.9.1}
		\subparagraph{Claim:} $\forall n \in \mathbb{N}$, 
			\[(x + y)^n = \sum_{k = 0}^n \binom{n}{k}x^{n-k}y^{k}\]
		\subparagraph{Proof:}
		Base case: When $n = 1$, $(x + y)^1 = x + y$ is obviously true. 
		Hypothesis: When $n = m$, \[(x + y)^m = \sum_{k = 0}^m \binom{m}{k}x^{m-k}y^{k}\]
		is true.
		
		Then when $k = m + 1$, 
		\begin{align}
			&(x + y)^{m + 1} = (x + y)^m \cdot (x + y)\nonumber\\
			&\phantom{(x + y)^{m + 1}} = (\sum_{k = 0}^{m} \binom{m}{k}x^{m - k}y^{k}) \cdot (x + y)\nonumber\\
			&\phantom{(x + y)^{m + 1}} = x \cdot \sum_{k = 0}^{m} \binom{m}{k}x^{m - k}y^{k} + y \cdot \sum_{j = 0}^{m} \binom{m}{j}x^{m - j}y^{j}\nonumber\\
			&\phantom{(x + y)^{m + 1}} = \sum_{k = 0}^{m} \binom{m}{k}x^{m + 1 - k}y^{k} + \sum_{j = 0}^{m} \binom{m}{j}x^{m - j}y^{j + 1}\nonumber\\
			&\phantom{(x + y)^{m + 1}} = \sum_{k = 0}^{m} \binom{m}{k}x^{m + 1 - k}y^{k} + \sum_{j = 0}^{m} \binom{m}{j + 1 - 1}x^{m + 1 - j - 1}y^{j + 1}\nonumber\\
			&\phantom{(x + y)^{m + 1}} = \sum_{k = 0}^{m} \binom{m}{k}x^{m + 1 - k}y^{k} + \sum_{k = 1}^{m} \binom{m}{k - 1}x^{m + 1 - k}y^{k}\nonumber\\
			&\phantom{(x + y)^{m + 1}} = \sum_{k = 0}^{m + 1} \binom{m}{k}x^{m + 1 - k}y^{k} - \binom{m}{m - 1} x^0y^k + \sum_{k = 0}^{m+1} \binom{m}{k-1}x^{m + 1 - k}y^{k} - \binom{m}{-1}x^{m+1}y^0\nonumber\\
			&\phantom{(x + y)^{m + 1}} = \sum_{k = 0}^{m+1}[\binom{m}{k} + \binom{m}{k - 1}]x^{m+1-k}y^k\nonumber\\
			&\phantom{(x + y)^{m + 1}} = \sum_{k = 0}^{m+1}\binom{m+1}{k}x^{m+1-k}y^k\nonumber
		\end{align}
		
		Thus, we can conclude that $\forall n \in \mathbb{N}$, 
			\[(x + y)^n = \sum_{k = 0}^n \binom{n}{k}x^{n-k}y^{k}\ \blacksquare\]
	\paragraph{1.10.2}
		\subparagraph{Claim:} $C_4 = \{i, -1, -i, 1\}$ is a group under complex multiplication.
		\subparagraph{Proof:}
			\begin{table}[!hbp]
    \centering
    \begin{tabular}{r || r | r | r | r}
        
        & $i$ & $-1$ & $-i$ & $1$\\
        \hline
        \hline
           
           $i$ & $-1$ & $-i$ & $1$ & $i$\\
        \hline
           $-1$ & $-i$ & $1$ & $i$ & $-1$\\
        \hline
           $-i$ & $1$ & $i$ & $-1$ & $-i$\\
		\hline
           $1$ & $i$ & $-1$ & $-i$ & $1$\\
    \end{tabular}
    \caption{Table of Multiplication of $C_4$}
\end{table}

	As the table above has shown, the set under complex multiplication has an identity $1$, and the inverses:
		\begin{itemize}
			\item $(i)^{-1} = -i$
			\item $(-1)^{-1} = -1$
			\item $(-i)^{-1} = i$
			\item $(1)^{-1} = 1$
		\end{itemize}
	And the closure is fulfilled and associativity is guaranteed by the associativity of $\mathbb{C}$.
	
	As a result, we can conclude that $C_4 = \{i, -1, -i, 1\}$ is a group under complex multiplication. $\blacksquare$
	\paragraph{1.10.4} 
		\subparagraph{Claim:} There exists at least 1 isomorphism between $C_4$ and $\mathbb{Z}_4$.
		\subparagraph{Proof:}
		It is easy to construct the following the bijections $f : C_4 \mapsto \mathbb{Z}_4$:
		\begin{itemize}
				\item $f(i) = [1]$
				\item $f(-1) = [2]$
				\item $f(-i) = [3]$
				\item $f(1) = [0]$
		\end{itemize}
		that make the multiplication tables match up.
		
		So it is true that there exists at least 1 isomorphism between $C_4$ and $\mathbb{Z}_4 \ \blacksquare$.
	\paragraph{1.10.9}
		\subparagraph{Claim:} Affine transformations $\mathrm{Aff} = \{f | f:\textbf{x}\mapsto S(\textbf{x}) + \textbf{b}\}$ forms a group under composition of maps.
		\subparagraph{Proof:} 
			\begin{itemize} 
				\item Closure:\\ Let $f = S_1(\textbf{x}) + \textbf{b}_1$ and $g = S_2(\textbf{x}) + \textbf{b}_2 \in \mathrm{Aff}$, then $f \circ g = S_1(S_2(\textbf{x}) + \textbf{b}_2) + \textbf{b}_1 = S_1(S_2(\textbf{x}) + \textbf{b}_2) + \textbf{b}_1 = S_1(S_2(\textbf{x})) + S_1(\textbf{b}_2) + \textbf{b}_1\in Aff$ which means any transformation composition form of arbitrary 2 affine transformations.
				\item Associativity:\\Let $f = S_1(\textbf{x}) + \textbf{b}_1$, $g = S_2(\textbf{x}) + \textbf{b}_2$,	$h = S_3(\textbf{x}) + \textbf{b}_3 \in T$	.
				 
					Then $f \circ g \circ h = (S_1(S_2(\textbf{x}) + \textbf{b}_2) + \textbf{b}_1) \circ (S_3(\textbf{x}) + \textbf{b}_3) = S_1(S_2(S_3(\textbf{x}) + \textbf{b}_3) + \textbf{b}_2) + \textbf{b}_1 = S_1(S_2(S_3(\textbf{x}))) + S_1(S_2(\textbf{b}_3)) + S_1(\textbf{b}_2) + \textbf{b}_1$.
					
					$f \circ (g \circ h) = (S_1(\textbf{x}) + \textbf{b}_1) \circ (S_2(S_3(\textbf{x}) + \textbf{b}_3) + \textbf{b}_2) = (S_1(\textbf{x}) + \textbf{b}_1) \circ (S_2(S_3(\textbf{x})) + S_2(\textbf{b}_3) + \textbf{b}_2) = S_1(S_2(S_3(\textbf{x}))) + S_1(S_2(\textbf{b}_3)) + S_1(\textbf{b}_2) + \textbf{b}_1$.
					
					Thus, $f \circ g \circ h = f \circ (g \circ h)$, which means its associativity is proved.
				\item Identity:\\It's obvious that $f(\textbf{x}) = \textbf{x}$ satisfy that $f \circ g = g$.
				\item Inverse:\\ The inverse of affine transformation $f^{-1}$ satisfy $f \circ f^{-1} = f^{-1} \circ = e$, so for $f(x) = S_1(x) + \textbf{b}_1, f^{-1}(\textbf{x}) = S_2(\textbf{x}) + \textbf{b}_2$, $f \circ f^{-1} = f^{-1} \circ = S_1(S_2(\textbf{x})) + S_1(\textbf{b}_2) + \textbf{b}_1$. To make this equal to identity, we need $S_1(S_2(\textbf{x})) = x$, and $S_1(\textbf{b}_2) + \textbf{b}_1 = 0$. Since $S_1, S_2$ are all invertible transformation, so the first condition can always be satisfied. And since the set of all n-dimension vectors form a group under addition, the second condition can be satisfied as well. Thus, for any affine transformation $f$, we can always find its inverse.
			\end{itemize}
			
			As a result, we can conclude that affine transformations $\mathrm{Aff} = \{f | f:\textbf{x}\mapsto S(\textbf{x}) + \textbf{b}\}$ forms a group under composition of maps $\blacksquare$.
	\paragraph{1.11.2}
		\subparagraph{Claim:} Let $K$ be any field, set $K[x]$ of polynomials with coefficients in $K$ form a commutative ring under usual addition and multiplication of polynomials. And the constant polynomial 1 is the multiplicative identity, and the only units are the constant polynomials.
		\subparagraph{Proof:} Prove it's a commutative ring first:
		
		 And to prove that this set is a commutative ring, we need to prove its an abelian group under addition first.
		 
		 Let $f, g, h$ be 2 arbitrary polynomials with coefficient in field $K$. 
		 
		 Since $K$ is a field, so we know that $\forall f \in K[x], x \in K, f(K) \in K$
		 
		 \paragraph{Closure:}

		 	Thus, the addition
		 	\[(f + g)(x) = f(x) + g(x) = \sum_{i = 0}^{n_2} (e_{1i} + e_{2i})x^i + \sum_{i = n_2 + 1}^{n_1} e_{1i}x^i\] 
		 	which is in $K[x]$, so the closure is proved.
		 	
		 \paragraph{Associativity:}
		 	
		 	Then
		 	\begin{align}
		 		&((f + g) + h)(x) = (f(x) + g(x)) + h(x)\nonumber\\
		 		&\phantom{((f + g) + h)(x)} = \sum_{i = 0}^{n_2} (e_{1i} + e_{2i})x^i + \sum_{i = n_2 + 1}^{n_1} (e_{1i} + e_{2i})x^i + \sum_{i = 0}^{n_3} e_{3i}x^i\nonumber\\
		 		&\phantom{((f + g) + h)(x)} = \sum_{i = 0}^{n_2} (e_{1i} + e_{2i} + e_{3i})x^i + \sum_{i = n_2 + 1}^{n_1} (e_{1i} + e_{2i})x^i + \sum_{i = 0}^{n_3} e_{3i}x^i\nonumber
			\end{align}
			
			\begin{align}
				&(f + (g + h)(x)) = f(x) + (g(x) + h(x))\nonumber\\
				&\phantom{(f + (g + h)(x))} = \sum_{i = 0}^{n_2}e_{1i}x^i + \sum_{i = 0}^{n_3} (e_{2i} + e_{3i})x^i + \sum_{i = n_3 + 1}^{n_2}e_{3i} \nonumber\\
				&\phantom{(f + (g + h)(x))} = \sum_{i = 0}^{n_2} (e_{1i} + e_{2i} + e_{3i})x^i + \sum_{i = n_2 + 1}^{n_1} (e_{1i} + e_{2i})x^i + \sum_{i = 0}^{n_3} e_{3i}x^i\nonumber
			\end{align}
			which means that $((f + g) + h)(x) = (f + (g + h)(x))$. So the associativity is proved.
			
			\paragraph{Commutativity:}
				\[f + g = \sum_{i = 0}^{n_2} (e_{1i} + e_{2i})x^i + \sum_{i = n_2 + 1}^{n_1} e_{1i}x^i  = g + f\]
			which means the commutativity is proved to be true.
			
			\paragraph{Identity:} $f(x) = 0$ fulfills the requirement that $0 + f = f + 0 = f$. So the identity is $f(x) = 0$.
			
			\paragraph{Inverse:} For an arbitrary polynomial $f$, we want to find a $f^{-1} = -f$ that $f + (-f) = 0$.\\
			
			
			As a result, we can conclude that $K[x]$ is an abelian group.\\
			
			Then we can start working on proving that $k[x]$ forms a commutative monoid under the multiplication.
			
			\paragraph{Closure:}
				\begin{align}
					&f \cdot g = (\sum_{i = 0}^{n_1} e_{1i}x^i)(\sum_{i = 0}^{n_2} e_{2i}x^i)\nonumber\\
					&\phantom{f \cdot g} = (e_{10}x^0 + e_{11}x^1 \cdots + e_{1n_1}^{n_1})\sum_{i = 0}^{n_2} e_{2i}x^i\nonumber\\
					&\phantom{f \cdot g} = \sum_{i = 0}^{n_2} (e_{10}x^0 + e_{11}x^1 \cdots + e_{1n_1}^{n_1})e_{2i}x^i\nonumber
				\end{align}
				
				Since $f(x) \in K$, we denote it as $e'$, then \[f \cdot g = \sum_{i = 0}^{n_2} (e'e_{2i})x^i \in K[x]\]
				Thus, the closure under multiplication is proved.
				
				\paragraph{Associativity}
				\begin{align}
					&(f \cdot g) \cdot h = (\sum_{j = 0}^{n_2} ((\sum_{i = 0}^{n_1} e_{1i}x^j)e_{2j})x^j) \cdot (\sum_{k = 0}^{n_3} e_{3j}x^k)\nonumber\\
					&\phantom{(f \cdot g) \cdot h} = \sum_{k = 0}^{n_3}(\sum_{j = 0}^{n_2} ((\sum_{i = 0}^{n_1} e_{1i}x^i)e_{2j})x^j)e_{3k})x^k\nonumber
				\end{align}
				
				\begin{align}
					&f \cdot (g \cdot h) = (\sum_{i = 0}^{n_1} e_{1i}x^i) \cdot (\sum_{k = 0}^{n_3} ((\sum_{j = 0}^{n_2} e_{2j}x^j)e_{2k})x^k)\nonumber\\
					&\phantom{(f \cdot g) \cdot h} = \sum_{k = 0}^{n_3} ((\sum_{i = 0}^{n_1} e_{1i}x^i)((\sum_{j = 0}^{n_2} e_{2j}x^j)e_{2k}))x^k\nonumber\\
					&\phantom{(f \cdot g) \cdot h} = \sum_{k = 0}^{n_3}(\sum_{j = 0}^{n_2} ((\sum_{i = 0}^{n_1} e_{1i}x^i)e_{2j})x^j)e_{3k})x^k\nonumber
				\end{align}
				
				Thus, $(f \cdot g) \cdot h = f \cdot (g \cdot h)$, and multiplication of $k[n]$ is associative.
				
				\paragraph{Identity:}
					Since for all $f$, we have $f \cdot 1 = 1 \cdot f = f$, its the identity of multiplication of $K[x]$
				\paragraph{Commutativity:}
					\[f \cdot g = \sum_{j = 0}^{n_2} ((\sum_{i = 0}^{n_1} e_{1i}x^j)e_{2j})x^j = g \cdot f\]
				Thus, the requirement of commutativity is fulfilled.
				
				Finally, we can work on the distributability of $+$ and $\cdot$ operations.
				
				\begin{align}
					&f(g + h) = (\sum_{i = 0}^{n_1} e_{1i}x^i)(\sum_{j = 0}^{n_3} (e_{2j} + e_{3j})x^j)\nonumber\\
					&\phantom{f(g + h)} = (\sum_{i = 0}^{n_1} e_{1i}x^i)(\sum_{j = 0}^{n_3} e_{2j} + \sum_{j = 0}^{n_3} e_{3j}x^j)\nonumber\\
					&\phantom{f(g + h)} = (\sum_{i = 0}^{n_1} e_{1i}x^i)(\sum_{j = 0}^{n_3} e_{2j}) + (\sum_{i = 0}^{n_1} e_{1i}x^i)(\sum_{j = 0}^{n_3} e_{3j}x^j)\nonumber\\
					&\phantom{f(g + h)} = fg + fh\nonumber
				\end{align}
				
				Similarly, $(f + g)h = fh + gh$, so the distributability is proved.\\
				
				And by definition, a unit is an element with a multiplicative inverse. Which is saying that for a unit $f$, there is a $f{-1}$ that $f \cdot f^{-1} = 1$. Since product of any 2 non-zero $x \in K$ will not be zero, the terms generated by the products of non-zero degree terms of 2 polynomials will not be cancelled out. As a result, only 0-degree polynomials, namely, constant polynomials are the only units.\\
				
				In conclusion, $K[x]$ of polynomials with coefficients in $K$ form a commutative ring under usual addition and multiplication of polynomials. And the constant polynomial 1 is the multiplicative identity and constant polynomials are the only units.$\blacksquare$
				
				
\end{document}
