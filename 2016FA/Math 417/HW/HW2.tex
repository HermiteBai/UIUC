\documentclass[11pt]{article}
\title{HW2}
%-----------Packeges---------------%
\usepackage{amsmath}
\usepackage{amssymb}
\usepackage{amsfonts}
\usepackage{tocloft}
\usepackage{float}
\usepackage{graphicx}
\usepackage[bookmarks=true]{hyperref}
\usepackage{fancyhdr}


%----------Definition & Theorem----%
\newtheorem{definition}{Definition}[subsection]
\newtheorem{theorem}{Theorem}[subsection]
\newtheorem{proposition}{Proposition}[subsection]
\newtheorem{lemma}{Lemma}[subsection]
\newtheorem{corollary}{Corollary}[subsection]

\pagestyle{fancy}
\fancyhead[L]{Math 417}
\fancyhead[C]{HW2}
\fancyhead[R]{Lanxiao Bai(lbai5)}

\begin{document}
\paragraph{1.5.3}\textbf{Solution:}
\subparagraph{(a)} \[\left( \begin{array}{ccccccc} 1 & 2 & 3 & 4 & 5 & 6 & 7\\ 2 & 5 & 6 & 3 & 7 & 4 & 1 \end{array} \right) = (1\ 2\ 5\ 7)(3\ 6\ 4)\]
\subparagraph{(b)} \[(1\ 2)(1\ 2\ 3\ 4\ 5) = (2\ 3\ 4\ 5)\]
\subparagraph{(g)} \[(1\ 2)(1\ 3)(1\ 4) = (4\ 3\ 2\ 1)\]
\subparagraph{(h)} \[(1\ 3)(1\ 2\ 3\ 4)(1\ 3) = (1\ 4)(3\ 2)\]

\paragraph{1.5.6}\textbf{Claim: } \[\left( \begin{array}{ccccccc} 1 & 2 & 3 & 4 & 5 & 6 & 7\\ 2 & 5 & 6 & 3 & 7 & 4 & 1 \end{array} \right)^{-1} = \left( \begin{array}{ccccccc} 1 & 2 & 3 & 4 & 5 & 6 & 7\\ 7 & 1 & 4 & 6 & 2 & 3 & 5 \end{array} \right)\]

\textbf{Solution:} To find the inverse of a permutation is to find the permutation that, when producted with the original one, get $e$. As a result, there's only need to check how we can put the disordered element back. Thus, 
    \[\left( \begin{array}{ccccccc} 1 & 2 & 3 & 4 & 5 & 6 & 7\\ 2 & 5 & 6 & 3 & 7 & 4 & 1 \end{array} \right)^{-1} = \left( \begin{array}{ccccccc} 1 & 2 & 3 & 4 & 5 & 6 & 7\\ 7 & 1 & 4 & 6 & 2 & 3 & 5 \end{array} \right)\]
\paragraph{1.5.11}\textbf{Solution:}
    \begin{itemize}
        \item When $size = 2$, $\pi = (1\ 2)$, $order = 2$
        \item When $size = 4$, $\pi = (1\ 2\ 4\ 3)$, $order = 4$
        \item When $size = 6$, $\pi = (1\ 2\ 4)(3\ 6\ 5)$, $order = 3$
        \item When $size = 8$, $\pi = (1\ 2\ 4\ 8\ 7\ 5)(3\ 6)$, $order = 6$
        \item When $size = 10$, $\pi = (1\ 2\ 4\ 8\ 5\ 10\ 9\ 7\ 3\ 6)$, $order = 10$
        \item When $size = 12$, $\pi = (1\ 2\ 4\ 8\ 3\ 6\ 12\ 11\ 9\ 5\ 10\ 7)$, $order = 12$
        \item When $size = 14$, $\pi = (1\ 2\ 4\ 8)(3\ 6\ 12\ 9)(5\ 10)(7\ 14\ 13\ 11)$, $order = 4$
        \item When $size = 16$, $\pi = (1\ 2\ 4\ 8\ 16\ 15\ 13\ 9)(3\ 6\ 12\ 7\ 14\ 11\ 5\ 10)$, $order = 8$
        \item When $size = 52$, $\pi = (1\ 2\ 4\ 8\ 16\ 32\ 11\ 22\ 44\ 35\ 17\ 34\ 15\ 30\ 7\ 14\ 28\ 3\ 6\ 12\ 24\ 48\ 43\ 33\ 13\ 26\\ 52\ 51\ 49\ 45\ 37\ 21\ 42\ 31\ 9\ 18\ 36\ 19\ 38\ 23\ 46\ 39\ 25\ 50\ 47\ 41\ 29\ 5\ 10\ 20\ 40\ 27)$, $order = 52$
    \end{itemize}

\paragraph{1.6.3}\textbf{Claim:} If $p \in \mathbb{N}$, $(p | ab \Rightarrow (p | a) \wedge (p | b)) \Rightarrow$ p is a prime.\\

\textbf{Proof: }
Suppose $p$ is not a prime, then it can be at least factorize into $p = f_1f_2$. Then it is easy to construct a $ab = pq$ that $p \nmid q$. Thus, $ab = f_1f_2q$ and it is likely that $a = f_1$ and $b = f_2q$, under which circumstace neither $a$ nor $b$ can be divided by $p$. As a result, $p$ must be a prime number to guarantee the validity of the property.
\end{document}
