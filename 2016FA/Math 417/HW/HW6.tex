\documentclass[11pt]{article}
%-----------Packeges---------------%
\usepackage{amsmath}
\usepackage{amssymb}
\usepackage{amsfonts}
\usepackage{tocloft}
\usepackage{float}
\usepackage{graphicx}
\usepackage[bookmarks=true]{hyperref}
\usepackage{fancyhdr}


%----------Definition & Theorem----%
\newtheorem{definition}{Definition}[subsection]
\newtheorem{theorem}{Theorem}[subsection]
\newtheorem{proposition}{Proposition}[subsection]
\newtheorem{lemma}{Lemma}[subsection]
\newtheorem{corollary}{Corollary}[subsection]

\pagestyle{fancy}
\fancyhead[L]{Math 417}
\fancyhead[C]{HW6}
\fancyhead[R]{Lanxiao Bai(lbai5)}
\begin{document}
	\paragraph{2.3.6}
		\subparagraph{Claim:} Let $S = <e, r^2, a\ |\ a^2 = e, (r^2)^3 = e, ar^2 = r^{-2}a> \subseteq D_6$, $S \cong D_3$.
		\subparagraph{Proof:} We can construct a map $\varphi: S \rightarrow D_3$ that $e_S \mapsto e_{D_3}, a_S \mapsto a_{D_3}, r^2_{S} \mapsto r_{D_3}$.
		
		It's not to list all the maps:
		\begin{itemize}
			\item $e_S \mapsto e_{D_3}$
			\item $r^2_{S} \mapsto r_{D_3}$
			\item $(r^2_{S})^2 \mapsto r^2_{D_3}$
			\item $(r^2_{S})^3 \mapsto r^3_{D_3}$
			\item $a_S \mapsto a_{D_3}$
			\item $a_Sr^2_{S} \mapsto a_{D_3}r_{D_3}$
			\item $a_S(r^2_{S})^2 \mapsto a_{D_3}r^2_{D_3}$
			\item $a_S(r^2_{S})^3 \mapsto a_{D_3}r^3_{D_3}$
		\end{itemize} 
		
		Since the whole map table was displayed above, it's obviously the map is injective and surjective, so it is bijective.
		
		Since the subgroup is generated by generator elements, we only need to check if these elements preserves structure.
		
		Since,
				
		\[\varphi(a^2_S) = \varphi(e_S) = e_{D_3} = \varphi(a_S)^2\]
		
		\[\varphi((r_S^2)^3) = \varphi(e_S) = e_{D_3} = r^3_{D_3} = \varphi(r_S^2)^3\]
		
		\[\varphi(a_Sr_S^2) = a_{D_3}r_{D_3} = \varphi(a_S)\varphi(r^2_S) = r_{D_3}^{-1}a_{D_3} = \varphi(r^{-2}_Sa_S) = \varphi(r^{-2}_S)\varphi(a_S)\]
		
		Thus, we can conclude that $S \cong D_3$.$\blacksquare$\\\\
	\paragraph{2.4.4}
		\subparagraph{Claim:}
			Let $\varphi: G \rightarrow H$ be a homomorphism of groups. $\forall B \subseteq H$, \[\varphi^{-1}(B) = \{g \in G: \varphi(g) \in B\}\]
			is a subgroup of G.
			
		\subparagraph{Proof:}
			Take subgroup $S_B \subseteq B$ where $\forall h \in S_B$, $\varphi^{-1}(h) \in G$ exists. Let $g_1, g_2 \in \varphi^{-1}(S_B)$. There exist elements $b_1, b_2 \in S_B$ that $h_i = \varphi^{-1}(b_i)$. Then \[g1g2 = \varphi^{-1}(b_1)\varphi^{-1}(b_2) = \varphi^{-1}(b_1b_2) \in \varphi^{-1}(S_B) \subseteq G,\]
			
			since $b_1b_2 \in S_B$. 
			
			Similarly, for $g \in \varphi^{-1}(S_B)$ there is an $b \in S_B$ that $\varphi^{-1}(b) = g$. Then we have 
					\[g^{-1} = (\varphi^{-1}(b))^{-1} = \varphi^{-1}(b^{-1}) \in \varphi^{-1}(S_B).\]
		 	Since identity $e_G$ is always in $\varphi^{-1}(S_B)$ so it's not empty.
		 	
		 	Thus, we can conclude that $\forall B \subseteq H$, \[\varphi^{-1}(B) = \{g \in G: \varphi(g) \in B\}\]
			is a subgroup of G.$\blacksquare$

	\paragraph{2.4.8}
		\subparagraph{Claim:} Let $\varphi: G \rightarrow H$ be a homomorphism of $G$ onto H. If $A$ is a normal subgroup of $G$, $\varphi(A)$ is a normal subgroup of H.
		\subparagraph{Proof:} Since $A$ is a normal subgroup of $G$, we know that $\forall g \in G$, then 
			\[gAg^{-1} = A\]
			
			We can take group homomorphism $\varphi$ from both sides and get 
			\begin{align}
				&\phantom{\Rightarrow}\varphi(gAg^{-1}) = \varphi(A)\nonumber\\
				&\Rightarrow\varphi(g)\varphi(A)\varphi(g^{-1}) = \varphi(A)
			\end{align}
			
			Since we know that $\varphi(g^{-1}) = (\varphi(g))^{-1}$, 
			\[(1) \Rightarrow \varphi(g)\varphi(A)\varphi(g)^{-1} = \varphi(A)\]
			
			Since $\varphi$ maps $G$ onto $H$, $\forall h \in H$, $\exists g \in G$ that $\varphi(g) = h$, which is saying that $\forall \varphi(g) \in H$, 
				\[\Rightarrow \varphi(g)\varphi(A)\varphi(g)^{-1} = \varphi(A)\] 
			is true.
			
			Thus, we can conclude that If $A$ is a normal subgroup of $G$, $\varphi(A)$ is a normal subgroup of H.$\blacksquare$
	\paragraph{2.4.13}
		\subparagraph{Claim:}
			Let $\varphi: \pi \rightarrow T(\pi)$, $\varphi$ is a group homomorphism from $S_n$ to $\mathrm{GL}(n, \mathbb{R})$. And $T$ is orthogonal matrix that has exactly one entry of 1 in each row and each column and 0 elsewhere.
		\subparagraph{Proof:}
			Let $\pi_1, \pi_2 \in S_n$, then
				\begin{align}
					&\varphi(\pi_1)\varphi(\pi_2) = \varphi((\pi_1(1), \pi_1(2), \cdots, \pi_1(n)))\varphi((\pi_2(1), \pi_2(2), \cdots, \pi_2(n)))\nonumber\\
					&\phantom{\varphi(\pi_1)\varphi(\pi_2)} = [\hat{e}_{\pi_1(1)}, \hat{e}_{\pi_1(2)}, \cdots \hat{e}_{\pi_1(n)}][\hat{e}_{\pi_2(1)}, \hat{e}_{\pi_2(2)}, \cdots \hat{e}_{\pi_2(n)}]\nonumber\\
					&\phantom{\varphi(\pi_1)\varphi(\pi_2)} = [\hat{e}_{\pi_2(\pi_1(1))}, \hat{e}_{\pi_2(\pi_1(2))}, \cdots \hat{e}_{\pi_2(\pi_1(n))}]\nonumber\\
					&\phantom{\varphi(\pi_1)\varphi(\pi_2)} = T(\pi_2(\pi_1))\nonumber\\
					&\phantom{\varphi(\pi_1)\varphi(\pi_2)} = \varphi(\pi_1\pi_2)\nonumber
				\end{align}
				Thus, $\varphi$ is a group homomorphism from $S_n$ to $\mathrm{GL}(n, \mathbb{R})$.
				
				Take $P \in T(\pi)$, for some permutation $\pi$ on $\{1,...,n\}$, $P_{ij} = d_{\pi(i)j}$, where $d_{kl} = 1$ if $k = l$ and $0$ otherwise. The $(i,j)$ entry of $PP^T$ is 
						\[\sum_{k = 1}^n P_{ik} P_{jk} = \sum_{k = 1}^n d_{\pi(i)k} d_{\pi(j)k} = d_{\pi(i)\pi(j)}\]. 

					Since $\pi$ is a permutation, $\pi(i) = \pi(j)$ if and only if $i = j$. Therefore $d_{\pi(i)\pi(j)} = d_{ij}$. It follows that $PP^T = I$. By a similar argument, $P^TP = I$. Hence, $P$ is orthogonal.$\blacksquare$ 
	\paragraph{2.5.4}\textbf{Solution:}
		We have $S_3 = \{e, (12), (13), (23), (123), (132)\}$.
		\subparagraph{(a)}
			Let $s \in S_3$, 
			\begin{align}
				&Hs = \{\{e, (12)\}, \{(13), (132)\}, \{(23), (123)\}\}\nonumber\\
				&sH = \{\{e, (12)\}, \{(13), (123)\}, \{(23), (132)\}\}\nonumber
			\end{align}
			
			Since $\{(13), (132)\} \neq \{(13), (123)\}$, we know that $sH \neq Hs$.
		\subparagraph{(b)}
			Let $s \in S_3$, 
			\begin{align}
				&sH = \{\{e,(123),(132)\},\{(12),(23),(13)\}\}\nonumber\\
				&Hs = \{\{e,(123),(132)\},\{(12),(23),(13)\}\}\nonumber
			\end{align}
			
			So we have $sH = Hs$.
\end{document}
