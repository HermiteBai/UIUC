\documentclass[11pt]{article}
%-----------Packeges---------------%
\usepackage{amsmath}
\usepackage{amssymb}
\usepackage{amsfonts}
\usepackage{tocloft}
\usepackage{float}
\usepackage{graphicx}
\usepackage[bookmarks=true]{hyperref}
\usepackage{fancyhdr}


%----------Definition & Theorem----%
\newtheorem{definition}{Definition}[subsection]
\newtheorem{theorem}{Theorem}[subsection]
\newtheorem{proposition}{Proposition}[subsection]
\newtheorem{lemma}{Lemma}[subsection]
\newtheorem{corollary}{Corollary}[subsection]

\pagestyle{fancy}
\fancyhead[L]{Math 414}
\fancyhead[C]{Lecture 2}
\fancyhead[R]{Lanxiao Bai}

\usepackage{listings}
\usepackage{color}
\usepackage{enumerate}

\definecolor{dkgreen}{rgb}{0,0.6,0}
\definecolor{gray}{rgb}{0.5,0.5,0.5}
\definecolor{mauve}{rgb}{0.58,0,0.82}

\lstset{frame=tb,
  language=Python,
  aboveskip=3mm,
  belowskip=3mm,
  showstringspaces=false,
  columns=flexible,
  basicstyle={\small\ttfamily},
  numbers=none,
  numberstyle=\tiny\color{gray},
  keywordstyle=\color{black},
  commentstyle=\color{dkgreen},
  stringstyle=\color{black},
  breaklines=true,
  breakatwhitespace=true,
  tabsize=3
}

\begin{document}
	\section{Compactness}
	We assume there are countably many sentence symbols.
	
	A finite set $\{\varphi_1, \cdots, \varphi_n\}$ of wff's is satisfiable iff $\varphi_1 \wedge \cdots \wedge \varphi_n$ is not a contradiction.
	
	We say that $\{\varphi_1, \cdots, \varphi_n\}$ is consistent when it's satisfiable.
	
	We say that $\mathcal{T}$ of wff's is consistent when $\mathcal{T}$ is finitely satisfiable.
	
	i.e. $\varphi_1 \wedge \cdots \wedge \varphi_n$ is not contradiction for any $\varphi_i \in \mathcal{T}$.
	
	Let $\mathcal{T}$ be a consistent set of wff's. We extend $\mathcal{T}$ to a maximal consistent set $\Delta$ of wff's.
	
	\begin{lemma}
		Let $S$ be a consistent set of wff's, and let $\varphi$ a wff. Then $S \cup \{\varphi\}$ or $S \cup \{\neg\varphi\}$ is consistent.
	\end{lemma}
	
	\begin{lemma}
		Let $S$ be a consistent set of wff's, then the following are equivalent:
		\begin{enumerate}
			\item $S$ is a maximal consistent set of wff's
			\item For any wff $\varphi$ either $\varphi \in S$ or $\neg\varphi \in S$.
		\end{enumerate}
	\end{lemma}
\end{document}