\documentclass[11pt]{article}
%-----------Packeges---------------%
\usepackage{amsmath}
\usepackage{amssymb}
\usepackage{amsfonts}
\usepackage{tocloft}
\usepackage{float}
\usepackage{graphicx}
\usepackage[bookmarks=true]{hyperref}
\usepackage{fancyhdr}


%----------Definition & Theorem----%
\newtheorem{definition}{Definition}[subsection]
\newtheorem{theorem}{Theorem}[subsection]
\newtheorem{proposition}{Proposition}[subsection]
\newtheorem{lemma}{Lemma}[subsection]
\newtheorem{corollary}{Corollary}[subsection]

\pagestyle{fancy}
\fancyhead[L]{Math 414}
\fancyhead[C]{Lecture 2}
\fancyhead[R]{Lanxiao Bai}

\usepackage{listings}
\usepackage{color}
\usepackage{enumerate}

\definecolor{dkgreen}{rgb}{0,0.6,0}
\definecolor{gray}{rgb}{0.5,0.5,0.5}
\definecolor{mauve}{rgb}{0.58,0,0.82}

\lstset{frame=tb,
  language=Python,
  aboveskip=3mm,
  belowskip=3mm,
  showstringspaces=false,
  columns=flexible,
  basicstyle={\small\ttfamily},
  numbers=none,
  numberstyle=\tiny\color{gray},
  keywordstyle=\color{black},
  commentstyle=\color{dkgreen},
  stringstyle=\color{black},
  breaklines=true,
  breakatwhitespace=true,
  tabsize=3
}

\begin{document}
	\section{Recall}
	A first order language $L$ is a set of formal symbols consisting of:
	\begin{itemize}
		\item Logical symbols: 
		\begin{itemize} 
			\item $\neg, \vee, \wedge, \rightarrow, \leftrightarrow, \forall, \exists$
			\item Parathesis: $(, )$
			\item Equality: $=$
		\end{itemize}
		\item Variables: $x, y, z,\cdots$
		\item k-ary relation symbols: $R, S, \cdots$
		\item k-ary function symbols: $f, g, h, \cdots$
		\item Constant symbols: $c, c'$
	\end{itemize}
	
	First order language can be uncountable, but we can usually take $L$ to be countable.
	
	An L-structure $\mathcal{M}$ is a nonempty $M$ together with 
	\begin{itemize}
		\item a k-ary relation $R^{\mathcal{M}}$ on $M$ for every k-ary relation symbol
		\item a k-ary function $f^{\mathcal{M}}$ on $M$ for every k-ary function symbol
		\item an element $c^{\mathcal{M}}$ for each constant symbol $c$
	\end{itemize}
	
	$\mathcal{M}$ is the structure.
	
	$M$ is the underlying set (domain) of $\mathcal{M}$.
	
	We also write 
	\[\mathcal{M} = (M; R^{\mathcal{M}})\]
	
	Ex. If $M = \mathbb{R}$ and $R^{\mathcal{M}} = \leq$, then we write $(\mathbb{R}; \leq)$
	
	\begin{definition}
		$\mathcal{M}$ is a symmetric L-structure if $xR^{\mathcal{M}}y$ iff $yR^{\mathcal{M}}x$ for all $x, y \in M$.
	\end{definition}
	
	\begin{definition}
		A partial order is a L-structure satisfying
		\begin{itemize}
			\item \[\forall x (xRx)\]
			\item \[\forall x \forall y [ (xRy) \wedge (yRx)] \rightarrow (x = y)\]
			\item \[\forall x \forall y \forall z [(xRy) \wedge (yRz)] \rightarrow (xRz)\]
		\end{itemize}
	\end{definition}
	
	\begin{definition}
		$R^{\mathcal{M}}$ is total order if 
		\[\forall x \forall y (x \leq y) \vee (y \leq x)\].
	\end{definition}
\end{document}