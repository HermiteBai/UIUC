\documentclass[11pt]{article}
%-----------Packeges---------------%
\usepackage{amsmath}
\usepackage{amssymb}
\usepackage{amsfonts}
\usepackage{tocloft}
\usepackage{float}
\usepackage{graphicx}
\usepackage[bookmarks=true]{hyperref}
\usepackage{fancyhdr}


%----------Definition & Theorem----%
\newtheorem{definition}{Definition}[subsection]
\newtheorem{theorem}{Theorem}[subsection]
\newtheorem{proposition}{Proposition}[subsection]
\newtheorem{lemma}{Lemma}[subsection]
\newtheorem{corollary}{Corollary}[subsection]

\pagestyle{fancy}
\fancyhead[L]{Math 414}
\fancyhead[C]{Lecture 2}
\fancyhead[R]{Lanxiao Bai}

\usepackage{listings}
\usepackage{color}
\usepackage{enumerate}

\definecolor{dkgreen}{rgb}{0,0.6,0}
\definecolor{gray}{rgb}{0.5,0.5,0.5}
\definecolor{mauve}{rgb}{0.58,0,0.82}

\lstset{frame=tb,
  language=Python,
  aboveskip=3mm,
  belowskip=3mm,
  showstringspaces=false,
  columns=flexible,
  basicstyle={\small\ttfamily},
  numbers=none,
  numberstyle=\tiny\color{gray},
  keywordstyle=\color{black},
  commentstyle=\color{dkgreen},
  stringstyle=\color{black},
  breaklines=true,
  breakatwhitespace=true,
  tabsize=3
}

\begin{document}
	\section{Theory}
	Let $\Sigma$ be a truth assignment, let $\Delta = \{\text{wff's that $\Sigma$ satisfies}\}$.
	
	$\Delta$ is trivially finitely satisfiable.
	
	$\Delta$ is maximal: $\varphi \in \Delta$ or $(\neg \varphi) \in \Delta$ for any wff.
	
	$\Delta$ is the theory of $\Sigma$.
	
	The truth assignment
	\[\Sigma(P) = 
	\begin{cases}
		T & \text{if } P \in \Delta\\
		T & \text{otherwise}
	\end{cases} \]
	
	satisfies $\Delta$, $\Delta$ is the theory of $\Sigma$.
	
	\section{First order logic}
	
	Let $M$ be a set, a k-nary on $M$ is a subset $R$ of $M^k$.
	
	We often write $R(x_1, x_2, \cdots, x_k)$ for $(x_1, x_2, \cdots, x_k) \in R$.
	
	If $R$ is a binary relation, we write $xRy$ when $R(x, y)$.
	
	A k-nary function on $M$ is a function $f: M^k \rightarrow M$.
	
	\subsection{First order language}
	
	A first order language $L$ is a set of formal symbols consisting of:
	\begin{itemize}
		\item Logical symbols: 
		\begin{itemize} 
			\item $\neg, \vee, \wedge, \rightarrow, \leftrightarrow, \forall, \exists$
			\item Parathesis: $(, )$
			\item Equality: $=$
		\end{itemize}
		\item Variables: $x, y, z,\cdots$
		\item k-ary relation symbols: $R, S, \cdots$
		\item k-ary function symbols: $f, g, h, \cdots$
		\item Constant symbols: $c, c'$
	\end{itemize}
	
	First order language can be uncountable, but we can usually take $L$ to be countable.
	
	An L-structure $\mathcal{M}$ is a nonempty $M$ together with 
	\begin{itemize}
		\item a k-ary relation $R^{\mathcal{M}}$ on $M$ for every k-ary relation symbol
		\item a k-ary function $f^{\mathcal{M}}$ on $M$ for every k-ary function symbol
		\item an element $c^{\mathcal{M}}$ for each constant symbol $c$
	\end{itemize}
	
	$\mathcal{M}$ is the structure.
	
	$M$ is the underlying set (domain) of $\mathcal{M}$.
	
	$\mathcal{M}$ is a symmetric L-structure if $xR^{\mathcal{M}y}$ iff $yR^{\mathcal{M}x}$ for all $x, y \in M$.
	
\end{document}