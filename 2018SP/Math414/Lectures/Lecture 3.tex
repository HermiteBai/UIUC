\documentclass[11pt]{article}
%-----------Packeges---------------%
\usepackage{amsmath}
\usepackage{amssymb}
\usepackage{amsfonts}
\usepackage{tocloft}
\usepackage{float}
\usepackage{graphicx}
\usepackage[bookmarks=true]{hyperref}
\usepackage{fancyhdr}


%----------Definition & Theorem----%
\newtheorem{definition}{Definition}[subsection]
\newtheorem{theorem}{Theorem}[subsection]
\newtheorem{proposition}{Proposition}[subsection]
\newtheorem{lemma}{Lemma}[subsection]
\newtheorem{corollary}{Corollary}[subsection]

\pagestyle{fancy}
\fancyhead[L]{Math 414}
\fancyhead[C]{Lecture 2}
\fancyhead[R]{Lanxiao Bai}

\usepackage{listings}
\usepackage{color}
\usepackage{enumerate}

\definecolor{dkgreen}{rgb}{0,0.6,0}
\definecolor{gray}{rgb}{0.5,0.5,0.5}
\definecolor{mauve}{rgb}{0.58,0,0.82}

\lstset{frame=tb,
  language=Python,
  aboveskip=3mm,
  belowskip=3mm,
  showstringspaces=false,
  columns=flexible,
  basicstyle={\small\ttfamily},
  numbers=none,
  numberstyle=\tiny\color{gray},
  keywordstyle=\color{black},
  commentstyle=\color{dkgreen},
  stringstyle=\color{black},
  breaklines=true,
  breakatwhitespace=true,
  tabsize=3
}

\begin{document}
	\section{Truth assignment}
	$S$ is the set of sentence symbols. A truth assignment is a function
	\[\Sigma : S \rightarrow \{T, F\} \]
	
	Given sets $A, B$, then $B^A$ is the set of functions
	\[f : A \rightarrow B\]
	
	so we may write 
	
	\[\{T, F\}^S\] for the sett of truth assignments.
	
	We identify $\{F, T\}$ with $\{0, 1\}$, and we often identify $\{0, 1\}$ with $2$, so we may write $2^S$ for the set of truth assignment.
	
	If there are $n$ sentence symbols, there are $2^n$ truth assignments.
	
	We can identify a truth assignment with the set $\{P \in S : \Sigma(P) = T\}$. Identify a subset $X \subseteq S$ with the truth assignment
	\[\Sigma(P) = 
	\begin{cases}
		T & P \in X\\
		F & P \notin X
	\end{cases} \]
	
	$S$ is the set of sentence symbols. Fix a truth assignment 
	\[\Sigma: \rightarrow \{T, F\}\]
	
	Extend $\Sigma$ to a function $\bar{\Sigma}$ from the set of wff's to $\{T, F\}$.
\end{document}