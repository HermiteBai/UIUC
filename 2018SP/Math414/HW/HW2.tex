\documentclass[11pt]{article}
%-----------Packeges---------------%
\usepackage{amsmath}
\usepackage{amssymb}
\usepackage{amsfonts}
\usepackage{tocloft}
\usepackage{float}
\usepackage{graphicx}
\usepackage[bookmarks=true]{hyperref}
\usepackage{fancyhdr}


%----------Definition & Theorem----%
\newtheorem{definition}{Definition}[subsection]
\newtheorem{theorem}{Theorem}[subsection]
\newtheorem{proposition}{Proposition}[subsection]
\newtheorem{lemma}{Lemma}[subsection]
\newtheorem{corollary}{Corollary}[subsection]

\pagestyle{fancy}
\fancyhead[L]{Math 414}
\fancyhead[C]{HW 2}
\fancyhead[R]{Lanxiao Bai}

\usepackage{listings}
\usepackage{color}
\usepackage{enumerate}

\newcommand{\rmodels}{%
  \mathrel{\text{\reflectbox{$\models$}}}%
}

\definecolor{dkgreen}{rgb}{0,0.6,0}
\definecolor{gray}{rgb}{0.5,0.5,0.5}
\definecolor{mauve}{rgb}{0.58,0,0.82}

\lstset{frame=tb,
  language=Python,
  aboveskip=3mm,
  belowskip=3mm,
  showstringspaces=false,
  columns=flexible,
  basicstyle={\small\ttfamily},
  numbers=none,
  numberstyle=\tiny\color{gray},
  keywordstyle=\color{black},
  commentstyle=\color{dkgreen},
  stringstyle=\color{black},
  breaklines=true,
  breakatwhitespace=true,
  tabsize=3
}

\begin{document}
	\paragraph{1.2.3}
	\begin{enumerate}[(a)]
		\item 
			 Suppose $\varphi = ((P \rightarrow Q) \vee (Q \rightarrow P))$
		
			 If $\tau: P = T, Q = T$, then $(P \rightarrow Q) = T$ and $(Q \rightarrow P) = T$, so $((P \rightarrow Q) \vee (Q \rightarrow P)) = T$.
		
			 If $\tau: P = T, Q = F$, then $(P \rightarrow Q) = F$ and $(Q \rightarrow P) = F$, $((P \rightarrow Q) \vee (Q \rightarrow P)) = T$.
			 
			 If $\tau: P = F, Q = T$, then $(P \rightarrow Q) = T$ and $(Q \rightarrow P) = F$, $((P \rightarrow Q) \vee (Q \rightarrow P)) = T$.
			 
			 If $\tau: P = F, Q = F$, then $(P \rightarrow Q) = T$ and $(Q \rightarrow P) = T$, $((P \rightarrow Q) \vee (Q \rightarrow P)) = T$.
			 
			 We see that for all truth assignment $\tau \in \varepsilon$, there is $\tau$ satisfies $\varphi$. As a result, $\varphi$ is a tautology.
			 
			 \item In order to satisfy $((P \wedge Q)\rightarrow R)$, we need truth assignments $v_1: P = T, Q = T, R = T; v_2: P = F, Q = T, R = T; v_3: P = F, Q =T, R = F; v_4: P = T, Q = F, R = T; v_5: P = T, Q = F, R = F; v_6: P = F, Q =F, R = T; v_7: P = F, Q = F, R = F$.
			 
			 	It is trivial to check that all these assignments satisfies $(P \rightarrow R) \vee (Q \rightarrow R)$.
	\end{enumerate}
	\paragraph{1.2.4}
	\begin{enumerate}[(a)]
		\item 
		\textbf{$\Rightarrow$:}
		Since $\Sigma; \alpha \models \beta$, $\alpha \models \beta$ then there is only two valid truth assignments $v_1: \alpha = T, \beta = T; v_2: \alpha = F, \beta = F$ with other assignments that satisfies every wff's in $\Sigma$. It is easy to check that both assignments will lead to $(\alpha \rightarrow \beta) = T$. Since we have that $\alpha, \beta \notin \Sigma$, whatever satisfies $\Sigma$ and $v_1, v_2$ should always have $\Sigma \models (\alpha \rightarrow \beta)$.
		
		\textbf{$\Leftarrow$:}
		Since we have $\Sigma \models (\alpha \rightarrow \beta)$, we have assignments that has $v_1: \alpha = T, \beta = T; v_2:  \alpha = F, \beta = T$ and other assignments that satisfies $\Sigma$. Then by Compactness Theorem, such assignments should also satisfy $\Sigma; \alpha \models \beta$.
		
		In conclusion, 
		\[\Sigma; \alpha \models \beta \text{ iff } \Sigma \models (\alpha \rightarrow \beta)\]
		
		\item
		\textbf{$\Rightarrow$:}
		Since $\alpha \models \rmodels \beta$, there is $\alpha \models \beta$ and $\beta \models \alpha$, so there are only two valid assignments $v_1: \alpha = T, \beta = T; v_2: \alpha = F, \beta = F$. In both cases, we have $\alpha \leftrightarrow \beta$. So that $\models (\alpha \leftrightarrow \beta)$.
		
		\textbf{$\Leftarrow$:}
		Since $\models (\alpha \leftrightarrow \beta)$, we can only have $v_1: \alpha = T, \beta = T; v_2: \alpha = F, \beta = F$ in the set of truth assignments. And it's trivial to check that for all these assignments, there is $\alpha \models \rmodels \beta$.
		
		In conclusion, $\alpha \models \rmodels \beta$ iff $\models (\alpha \leftrightarrow \beta)$.
		\end{enumerate}
		
		\paragraph{1.2.7}
		The question would be: If I ask you if the right road leads to the capital, would you say yes?
		
		Then if the person always tells the truth, he will confirm the correct answer but negate the wrong answer.
		
		If the person always tells the lie, he would lie, therefore, he will lie again about his own false answer, reporting the truth.
		
		\paragraph{1.2.9}
		\textbf{Proof:} 
		
		Base case: When $\alpha$ is a sentence symbol, then $\alpha^* = \neg \alpha$, which means $\alpha^* \models \rmodels \neg \alpha$.
		
		Suppose for a wff $\alpha$, $\alpha^* \models \rmodels \neg \alpha$ holds.
		
		Then $\alpha = A \vee B$, then $\alpha^* = (\neg A) \wedge (\neg B) = \neg (A \vee B) = \neg \alpha$.
		
		Similarly, $\alpha = A \wedge B$, then $\alpha^* = (\neg A) \vee (\neg B) = \neg (A \wedge B) = \neg \alpha$.
		
		In conclusion, by the Induction Principle, $\alpha^* \models \rmodels \neg \alpha$ holds for all wff $\alpha$.
		
		\paragraph{1.2.13}
		Let $P = $ "Playing tennis", $W = $ "Watching tennis", $R = $ "Reading about tennis".
		
		Then the statement of the speaker is: $$((\neg P) \rightarrow W) \wedge ((\neg W) \rightarrow R)$$
		
		By assumption, we know there should be exactly one among $P, W, R$ to be $T$, so our only valid assignment is $P = F, W = T, R = F$. So that we know the speaker is watching tennis.
\end{document}