\documentclass[11pt]{article}
%-----------Packeges---------------%
\usepackage{amsmath}
\usepackage{amssymb}
\usepackage{amsfonts}
\usepackage{tocloft}
\usepackage{float}
\usepackage{graphicx}
\usepackage[bookmarks=true]{hyperref}
\usepackage{fancyhdr}


%----------Definition & Theorem----%
\newtheorem{definition}{Definition}[subsection]
\newtheorem{theorem}{Theorem}[subsection]
\newtheorem{proposition}{Proposition}[subsection]
\newtheorem{lemma}{Lemma}[subsection]
\newtheorem{corollary}{Corollary}[subsection]

\pagestyle{fancy}
\fancyhead[L]{Math 414}
\fancyhead[C]{HW 3}
\fancyhead[R]{Lanxiao Bai}

\usepackage{listings}
\usepackage{color}
\usepackage{enumerate}

\newcommand{\rmodels}{%
  \mathrel{\text{\reflectbox{$\models$}}}%
}

\definecolor{dkgreen}{rgb}{0,0.6,0}
\definecolor{gray}{rgb}{0.5,0.5,0.5}
\definecolor{mauve}{rgb}{0.58,0,0.82}

\lstset{frame=tb,
  language=Python,
  aboveskip=3mm,
  belowskip=3mm,
  showstringspaces=false,
  columns=flexible,
  basicstyle={\small\ttfamily},
  numbers=none,
  numberstyle=\tiny\color{gray},
  keywordstyle=\color{black},
  commentstyle=\color{dkgreen},
  stringstyle=\color{black},
  breaklines=true,
  breakatwhitespace=true,
  tabsize=3
}

\begin{document}
	\begin{enumerate}
		\item 
		\textbf{Proof:}
		
		Suppose there's a function $g$ that there is no wff $\varphi$ such that $g = f_{\varphi}$. Then for any wff $\varphi$, whenever $g = T$, $\Sigma(\varphi) = F$. But if $g = x_1$, $\varphi = P_1$ and $\Sigma(P_1) = x_1 = T$, this truth assignment will lead to $g = T = \Sigma_{P_1}$. This fact contradicts with our assumption. 
		
		As a result, for all function $g$ there is a wff $\varphi$ that $g = f_{\varphi}$.$\blacksquare$
		
		\item
		\textbf{Proof:}
		
		\textbf{$\Rightarrow$:}
		
		Since $f_{\varphi} = f_{\psi}$, any truth assignment $\Sigma$ that either satisfies $(\varphi \wedge \psi)$ or does not satisfy $(\varphi \vee \psi)$. So by definition, $\Sigma$ either satisfies both $\varphi$ and satisfies $\psi$, or does not satisfies $\varphi$ and does not satisfies $\psi$. 
		
		This means that for all truth assignment, there is $\Sigma(\varphi) = \Sigma(\psi)$, so $(\varphi \leftrightarrow \psi)$ is always true.
		
		By definition, $(\varphi \leftrightarrow \psi)$ is a tautology.
		
		\textbf{$\Leftarrow$:}
		
		Since $(\varphi \leftrightarrow \psi)$ is a tautology, then for any truth assignment $\Sigma$, there is $\Sigma(\varphi) = \Sigma(\psi)$, so that $\Sigma$ that either satisfies $(\varphi \wedge \psi)$ or does not satisfy $(\varphi \vee \psi)$.
		
		As a result, $f_{\varphi} = f_{\psi}$.
		
		$\blacksquare$

		\item By counting the number of classes, we can calculate how many different mappings from $\{T, F\}^n$ to $\{T, F\}$.
		 
		 From the knowledge from Analysis, we know the number is 
		 \[|\{T,F\}|^{|\{T,F\}|^n} = 2^{2^n}\]
	\end{enumerate}
\end{document}