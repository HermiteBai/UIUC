\documentclass[11pt]{article}
%-----------Packeges---------------%
\usepackage{amsmath}
\usepackage{amssymb}
\usepackage{amsfonts}
\usepackage{tocloft}
\usepackage{float}
\usepackage{graphicx}
\usepackage[bookmarks=true]{hyperref}
\usepackage{fancyhdr}


%----------Definition & Theorem----%
\newtheorem{definition}{Definition}[subsection]
\newtheorem{theorem}{Theorem}[subsection]
\newtheorem{proposition}{Proposition}[subsection]
\newtheorem{lemma}{Lemma}[subsection]
\newtheorem{corollary}{Corollary}[subsection]

\pagestyle{fancy}
\fancyhead[L]{Math 414}
\fancyhead[C]{HW 5}
\fancyhead[R]{Lanxiao Bai}

\usepackage{listings}
\usepackage{color}
\usepackage{enumerate}

\newcommand{\rmodels}{%
  \mathrel{\text{\reflectbox{$\models$}}}%
}

\definecolor{dkgreen}{rgb}{0,0.6,0}
\definecolor{gray}{rgb}{0.5,0.5,0.5}
\definecolor{mauve}{rgb}{0.58,0,0.82}

\lstset{frame=tb,
  language=Python,
  aboveskip=3mm,
  belowskip=3mm,
  showstringspaces=false,
  columns=flexible,
  basicstyle={\small\ttfamily},
  numbers=none,
  numberstyle=\tiny\color{gray},
  keywordstyle=\color{black},
  commentstyle=\color{dkgreen},
  stringstyle=\color{black},
  breaklines=true,
  breakatwhitespace=true,
  tabsize=3
}

\begin{document}
	\begin{enumerate}
		\item Note that a automorphism should be able to preserve the structure of the original algebraic structure, which means that if $g_{\lambda}$ is an automorphism, then we should have for all $x \leq y \in \mathbb{R}$, there is 
		\[g_{\lambda}(x) \leq g_{\lambda}(y) \Rightarrow \lambda x \leq \lambda y \Rightarrow \lambda (x - y) \leq 0\]
		
		Since $x \leq y$, $x - y \leq 0$, so in order to make $g_{\lambda}$ an automorphism, it is required that $\lambda \geq 0$.
		\item 
		\begin{enumerate}
			\item Since for all $x \in \mathbb{Z}$ we have $-x$ that $g(-x) = x \in \mathbb{Z}$, and for all $x, y \in \mathbb{Z}, (x \neq y) \rightarrow g(x) \neq g(y)$ so we see that $g$ is a bijection between $\mathbb{Z}$ and $\mathbb{Z}$.
			
			We also see that for all $x, y \in \mathbb{Z}$, $g(0) = 0$ and 
			\[g(x + y) = -(x + y) = (-x) + (-y) = g(x) + g(y)\]
			
			so $g$ is a homomorphism.
			
			In conclusion, $g$ is an automorphism of $(\mathbb{Z}, +,  0)$.
			\item There is no other automorphism.
			
			\textbf{Proof:}
			
			In order to preserve the structure, any mapping $h$ and $m \in \mathbb{Z}$ needs to have
			\[h(m) = h(\underbrace{1 + 1 + \cdots + 1}_{m}) = \underbrace{h(1) + h(1) + \cdots + h(1)}_{m} = mh(1)\]
			
			In order to make this mapping surjective, we see that $h(1) = 1$ or $h(1) = -1$, which means that there are no other valid automorphisms for $(\mathbb{Z}, +,  0)$.
		\end{enumerate}
		\item Similarly, in order to preserve the structure, any mapping $h$ and $m \in \mathbb{N}$ needs to have
		
			\[h(m) = mh(1)\]
			
			In order to make this mapping bijective, it is only possible to let $h(x) = x$. So the identity is the only possible automorphism.
		\item \textbf{Proof:}
			We first prove $(\mathcal{P}(X), \oplus, 0)$ is an Abelian group.
			
			\begin{itemize}
				\item \textbf{Closure}
					For any $x, y \in \mathcal{P}(X)$, there is 
				\[x \oplus y = x \triangle y \subseteq x \cup y\]
				
				which has $x \oplus y \subseteq x \cup y \subseteq X$, so $x \oplus y \in \mathcal{P}(X)$.
				
				\item \textbf{Commutivity}
					By set theory we know for any $x, y \in \mathcal{P}(X)$
					\[x \oplus y = x \triangle y = y \triangle x = y \oplus x\]
					holds.
				\item \textbf{Associativity}
					By set theory we know for any $x, y, z \in \mathcal{P}(X)$
					\[x \oplus y \oplus z = x \triangle y \triangle z = x \triangle (y \triangle z) = x \oplus (y \oplus z)\]
					holds.
				\item \textbf{Additive identity}
					For any $x \in \mathcal{P}(X)$
					\[x \oplus 0 = x \triangle \emptyset = x\]
				
				\item \textbf{Additive inverse}
					\[x \oplus x = x \triangle x = \emptyset = 0\]
			\end{itemize}
		
		Then we want to prove $(\mathcal{P}(X), \otimes, 1)$ is a monoid
			\begin{itemize}
				\item \textbf{Closure}
				For any $x, y \in \mathcal{P}(X)$, there is 
				\[x \otimes y = x \cap y\]
				which has $x \otimes y \subseteq x \subseteq X$ and $x \otimes y \subseteq y \subseteq X$, so $x \otimes y \in \mathcal{P}(X)$
				
				\item \textbf{Associativity}
				By set theory we know for any $x, y, z \in \mathcal{P}(X)$
					\[x \otimes y \otimes z = x \cap y \cap z = x \cap (y \cap z) = x \otimes (y \otimes z)\]
					holds.
				\item \textbf{Multiplicative identity}
				For any $x \in \mathcal{P}(X)$
				\[x \otimes 1 = x \cap X = x\]
			\end{itemize}
			
		In the last, we want to prove the Distributivity.
		For any $x, y, z \in \mathcal{P}(X)$, 
		\begin{align}
			&x \otimes (y \oplus z) = x \cap (y \triangle z)\nonumber\\
			&\phantom{x \otimes (y \oplus z)} = (x \cap y) \triangle (x \cap z)\nonumber\\
			&\phantom{x \otimes (y \oplus z)} = (x \otimes y) \oplus (x \otimes z)\nonumber
		\end{align}
		
		\begin{align}
			&(y \oplus z) \otimes x = (y \triangle z) \cap x\nonumber\\
			&\phantom{x \otimes (y \oplus z)} = (y \cap x) \triangle (z \cap x)\nonumber\\
			&\phantom{x \otimes (y \oplus z)} = (y \otimes x) \oplus (z \otimes x)\nonumber
		\end{align}
		
		In conclusion, we proved that $(\mathcal{P}(X), \oplus, \otimes, 0, 1)$ is a ring.
	\end{enumerate}
\end{document}