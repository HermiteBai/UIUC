\documentclass[11pt]{article}
%-----------Packeges---------------%
\usepackage{amsmath}
\usepackage{amssymb}
\usepackage{amsfonts}
\usepackage{tocloft}
\usepackage{float}
\usepackage{graphicx}
\usepackage[bookmarks=true]{hyperref}
\usepackage{fancyhdr}


%----------Definition & Theorem----%
\newtheorem{definition}{Definition}[subsection]
\newtheorem{theorem}{Theorem}[subsection]
\newtheorem{proposition}{Proposition}[subsection]
\newtheorem{lemma}{Lemma}[subsection]
\newtheorem{corollary}{Corollary}[subsection]

\pagestyle{fancy}
\fancyhead[L]{Math 414}
\fancyhead[C]{HW 1}
\fancyhead[R]{Lanxiao Bai}

\usepackage{listings}
\usepackage{color}
\usepackage{enumerate}


\definecolor{dkgreen}{rgb}{0,0.6,0}
\definecolor{gray}{rgb}{0.5,0.5,0.5}
\definecolor{mauve}{rgb}{0.58,0,0.82}

\lstset{frame=tb,
  language=Python,
  aboveskip=3mm,
  belowskip=3mm,
  showstringspaces=false,
  columns=flexible,
  basicstyle={\small\ttfamily},
  numbers=none,
  numberstyle=\tiny\color{gray},
  keywordstyle=\color{black},
  commentstyle=\color{dkgreen},
  stringstyle=\color{black},
  breaklines=true,
  breakatwhitespace=true,
  tabsize=3
}

\begin{document}
	\begin{enumerate}
		\item 
		\begin{enumerate}
			\item Suppose we have $a, b, c \in \mathbb{Z}$.
			
			Since $a - a = 0$ which is even, we have $a \equiv a$ (mod 2), so reflexivity is proved.
			
			Suppose $a \equiv b$ (mod 2), then we have $k \in \mathbb{Z}$ that $a - b = 2k$ so that $b - a = -2k$ and $-2k \in \mathbb{Z}$. So we have $b \equiv a$ (mod 2) and thus the symmetry is proved.
			
			Suppose $a \equiv b$ (mod 2) and  $b \equiv c$ (mod 2), so we have $k_1, k_2 \in \mathbb{Z}$ that $a - b = 2k_1$, $b - c = 2k_2$. So $a - c = (a - b) - (b - c) = 2(k_1 - k_2)$ that $k_1 - k_2 \in \mathbb{Z}$. Thus, $a \equiv c$ (mod 2). So transitivity is proved.
			
			In conclusion, this relation is equivalent.
			$\blacksquare$
			
			\item Since we have $m \equiv n$ (mod 2) and $m' \equiv n'$ (mod 2), so let $k_1, k_2 \in \mathbb{Z}$ that $m - n = 2k_1$ and $m' - n' = 2k_2$.
			
			As a result, 
			\begin{align}
				&(m + n) - (m' + n') = (m - m') + (n - n')\nonumber\\ 
				&\phantom{(m + n) - (m' + n')} = 2(n - n') + 2(k_1 - k_2)\nonumber\\
				&\phantom{(m + n) - (m' + n')} = 2(n - n' + k_1 - k_2)\nonumber
			\end{align}
			
			Since $n, n', k_1, k_2 \in \mathbb{Z}$, $n - n' + k_1 - k_2 \in \mathbb{Z}$. So we have $m + n \equiv m' + n'$ (mod 2).
			
			Similarly, 
			\begin{align}
				&mm' - nn' = (n + 2k_1)(n' + 2k_2) - nn'\nonumber\\
				&\phantom{mm' - nn'} = nn' + 2k_1n' + 2k_2n + 4k_1k_2 - nn'\nonumber\\
				&\phantom{mm' - nn'} = 2k_1n' + 2k_2n + 4k_1k_2\nonumber
			\end{align}
			
			So we proved that $mm' \equiv nn'$ (mod 2).
			$\blacksquare$
			\item For all $x \in U$, if $x \in A$ and $x \in B$ then $x \in A \cap B$ so that
			\[\chi_{A \cap B}(x) = 1 = \chi_{A}(x)\chi_{B}(x)\]
			
			if $x \notin A$ or $x \notin B$, then
			\[\chi_{A \cap B}(x) = 0 = \chi_{A}(x)\chi_{B}(x)\]
			
			so we have \[\chi_{A \cap B}(x) \equiv \chi_{A}(x)\chi_{B}(x) (mod\ 2)\]
			
			Suppose $x \in A \Delta B$, then $x \in A$ xor $x \in B$ so
			\[\chi_{A \Delta B}(x) = 1 = 1 + 0 = \chi_{A}(x) + \chi_{B}(x)\]
			
			Suppose $x \notin A \Delta B$, then $x \notin B$ and $x \notin A$, then so
			\[\chi_{A \Delta B}(x) = 0 = 0 + 0 = \chi_{A}(x) + \chi_{B}(x)\]
			
			or $x \notin A \Delta B$, then $x \in B$ and $x \in A$, then so
			\[\chi_{A \Delta B}(x) = 0\]
			
			\[\chi_{A}(x) + \chi_{B}(x) = 1 + 1 = 2\]
			 
			 So we have \[\chi_{A \Delta B}(x) \equiv \chi_{A}(x) + \chi_{B}(x) (mod\ 2)\]
			 $\blacksquare$
		\end{enumerate}
		\item 
		\begin{enumerate}[(1).]
			\item Since $f, g$ injective, for all $x_1, x_2 \in A$, $x_1 \neq x_2$ implies $f(x_1) \neq f(x_2)$ so $g(f(x_1)) \neq g(f(x_2))$, so we conclude that $g \circ f$ is injectives.
			\item Since $f, g$ surjective for all $z \in C$ there is $y \in B$ that $g(y) = z$ and for all $y \in B$ there is $x \in A$ that $f(x) = y$. So for all $z \in C$ there is $x \in A$ that $g(f(x)) = z$. We conclude that $g \circ f$ is surjective.
			\item Suppose $f$ is not injective then there exist $x_1, x_2 \in A$ that $f(x_1) = f(x_2)$ so $g(f(x_1)) = g(f(x_2))$ which means $g \circ f$ is not injective. So we conclude that $f$ must be injective.
			\item Suppose $g \circ f(x) = x$, $f(x) = \sqrt (x)$ and $g(x) = x^2$, we see that $g \circ f$ is injective but not $g$. So we conclude that $g$ is not necessarily injective.
			\item Let $A = \{0\}, B = \{0, 1\}$ and $f(0) = 0, g(x) = 0$, so $g \circ f$ is surjective, but $f$ is not surjective.
			\item If $g$ is not surjective, then there is $z \in C$ that no $y \in B$ that $g(y) = z$. Since for all $x \in A$, $f(x) \in B$. So there is no $x \in A$ that $g(f(x)) = z$. So we conclude that $g$ must be surjective.
		\end{enumerate}
		\item 
			\begin{enumerate}
				\item 
				\[f(x) = 
				\begin{cases}
					x / 2 & \text{if $x$ is even}\\
					-\frac{x + 1}{2} & \text{otherwise}
				\end{cases} \]
				\item Suppose we have countable sets $A, B$, then we can construct a bijection between $A$ and odd numbers and between $B$ and even numbers. Then we see that there is a bijection between the union of $A$ and $B$ to the union of odd numbers and between $B$ and even numbers, which is $\mathbb{Z}$. So we conclude that the union of two countable sets is countable.
			\end{enumerate}
			\item Base case: When $k = 1$, $A^k = A$, which has $1^k = 1$ element.
			
			Suppose when $k = m$, $|A^k| = n^m$.
			
			Then when $k = m + 1$, $|A^{m + 1}| = |A\times A^m| = |A||A^m| = n\cdot n^m = n^{m + 1}$.
			
			In conclusion, $|A^k| = n^k$ for all $k \geq 1$.
			$\blacksquare$
			\item
			\begin{enumerate}[(1).]
				\item
				\begin{align}
					&\phantom{\Leftrightarrow}V \in \mathcal{P}(A) \cap \mathcal{P}(B)\nonumber\\
					&\Leftrightarrow \mathcal{P}(A) \vee V \in \mathcal{P}(B)\nonumber\\
					&\Leftrightarrow V \subseteq A \vee V \subseteq B\nonumber\\
					&\Leftrightarrow V \subseteq A \cap \subseteq B\nonumber\\
					&\Leftrightarrow V \in \mathcal{P}(A \cap B)\nonumber
				\end{align}
				
				So $\mathcal{P}(A) \cap \mathcal{P}(B) = \mathcal{P}(A \cap B)$.
				\item Let $A = \{1, 2\}$, $B = \{2, 3\}$, so $A \cap B = \{1, 2, 3\}$. 
				\item 
				Note that $A \cap B = \{1, 2, 3\} \not\subseteq A$ and $A \cap B = \{1, 2, 3\} \not\subseteq B$.

				So $A \cap B \not\in \mathcal{P}(A) \wedge A \cap B \not\in \mathcal{P}(B)$.
				
				Hence, $A \cap B \not\in \mathcal{P}(A) \cup \mathcal{P}(B)$.
				
				So $\mathcal{P}(A \cup B) \not\subseteq \mathcal{P}(A) \cup \mathcal{P}(B)$.
				
				We conclude that this is not necessarily correct.
			\end{enumerate}
	\end{enumerate}
\end{document}