\documentclass[11pt]{article}
%-----------Packeges---------------%
\usepackage{amsmath}
\usepackage{amssymb}
\usepackage{amsfonts}
\usepackage{tocloft}
\usepackage{float}
\usepackage{graphicx}
\usepackage[bookmarks=true]{hyperref}
\usepackage{fancyhdr}


%----------Definition & Theorem----%
\newtheorem{definition}{Definition}[subsection]
\newtheorem{theorem}{Theorem}[subsection]
\newtheorem{proposition}{Proposition}[subsection]
\newtheorem{lemma}{Lemma}[subsection]
\newtheorem{corollary}{Corollary}[subsection]

\pagestyle{fancy}
\fancyhead[L]{Math 414}
\fancyhead[C]{HW 4}
\fancyhead[R]{Lanxiao Bai}

\usepackage{listings}
\usepackage{color}
\usepackage{enumerate}

\newcommand{\rmodels}{%
  \mathrel{\text{\reflectbox{$\models$}}}%
}

\definecolor{dkgreen}{rgb}{0,0.6,0}
\definecolor{gray}{rgb}{0.5,0.5,0.5}
\definecolor{mauve}{rgb}{0.58,0,0.82}

\lstset{frame=tb,
  language=Python,
  aboveskip=3mm,
  belowskip=3mm,
  showstringspaces=false,
  columns=flexible,
  basicstyle={\small\ttfamily},
  numbers=none,
  numberstyle=\tiny\color{gray},
  keywordstyle=\color{black},
  commentstyle=\color{dkgreen},
  stringstyle=\color{black},
  breaklines=true,
  breakatwhitespace=true,
  tabsize=3
}

\begin{document}
	\begin{enumerate}[(1)]
		\item Suppose $\Sigma$ satisfies $(\neg \alpha)$, then $\Sigma$ does not satisfy $\alpha$, so $\alpha \notin \Delta$. Since $\Delta$ is a finitely satisfiable set, $(\neg \alpha) \in \Delta$.
		
			Suppose $(\neg \alpha) \in \Delta$, then $\alpha \notin \Delta$, so $\Sigma$ does not satisfy $\alpha$. Thus, $\Sigma$ satisfies $(\neg \alpha)$.
			
			In conclusion, $\Sigma$ satisfies $(\neg \alpha)$ iff $(\neg \alpha) \in \Delta$.
			
			$\blacksquare$
			
		\item Suppose $\Sigma$ satisfies $(\alpha \wedge \beta)$ then $\Sigma$ satisfies $\alpha$ and $\Sigma$ satisfies $\beta$. So $\alpha \in \Delta$ and $\beta \in \Delta$, which by definition means $(\alpha \wedge \beta) \in \Delta$.
		
			Suppose $(\alpha \wedge \beta) \in \Delta$, then $\alpha \in \Delta$ and $\beta \in \Delta$, so $\Sigma$ satisfies $\alpha$ and $\Sigma$ satisfies $\beta$, which by definition means $\Sigma$ satisfies $(\alpha \wedge \beta)$.
			
			In conclusion, $\Sigma$ satisfies $(\alpha \wedge \beta)$ iff $(\alpha \wedge \beta) \in \Delta$.
			
			$\blacksquare$
			
			\item Suppose $\Sigma$ satisfies $(\alpha \vee \beta)$ then $\Sigma$ satisfies $\alpha$ or $\Sigma$ satisfies $\beta$. So $\alpha \in \Delta$ or $\beta \in \Delta$, which by definition means $(\alpha \vee \beta) \in \Delta$.
		
			Suppose $(\alpha \vee \beta) \in \Delta$, then $\alpha \in \Delta$ or $\beta \in \Delta$, so $\Sigma$ satisfies $\alpha$ or $\Sigma$ satisfies $\beta$, which by definition means $\Sigma$ satisfies $(\alpha \vee \beta)$.
			
			In conclusion, $\Sigma$ satisfies $(\alpha \vee \beta)$ iff $(\alpha \vee \beta) \in \Delta$.
			
			$\blacksquare$
			
			\item Suppose $\Sigma$ satisfies $(\alpha \rightarrow \beta)$, then $\Sigma$ satisfies $(\neg \alpha)$ or $\Sigma$ satisfies $\beta$. So by what we has proved above, $(\neg\alpha) \in \Delta$ or $\beta \in \Delta$. So by definition, $(\alpha \rightarrow \beta) \in \Delta$.
			
				Suppose $(\alpha \rightarrow \beta) \in \Delta$, then $(\neg\alpha) \in \Delta$ or $\beta \in \Delta$, so by what we has proved above, $\Sigma$ satisfies $(\neg \alpha)$ or $\Sigma$ satisfies $\beta$. Thus, $\Sigma$ satisfies $(\alpha \rightarrow \beta)$.
				
				In conclusion, $\Sigma$ satisfies $(\alpha \rightarrow \beta)$ iff $(\alpha \rightarrow \beta) \in \Delta$.
			\item Suppose $\Sigma$ satisfies $(\alpha \leftrightarrow \beta)$ then $\Sigma$ satisfies $(\alpha \rightarrow \beta)$ and $(\beta \rightarrow \alpha)$, so $(\alpha \rightarrow \beta) \in \Delta$ and $(\beta \rightarrow \alpha) \in \Delta$. Thus, $(\alpha \leftrightarrow \beta) \in \Delta$.
			
			Suppose $(\alpha \leftrightarrow \beta) \in \Delta$, then $(\alpha \rightarrow \beta) \in \Delta$ and $(\beta \rightarrow \alpha) \in \Delta$, so $\Sigma$ satisfies $(\alpha \rightarrow \beta)$ and $(\beta \rightarrow \alpha)$. Thus, $\Sigma$ satisfies $(\alpha \leftrightarrow \beta)$.
			
			In conclusion, $\Sigma$ satisfies $(\alpha \leftrightarrow \beta)$ iff $(\alpha \leftrightarrow \beta) \in \Delta$.
			
			$\blacksquare$
			
			\item [(7)]
			
			Base case: Suppose $\varphi$ is a sentence symbol, $\varphi$ is a tautology iff any $\Sigma$ satisfy $\varphi$ iff $\varphi \in \Delta$.
			
			Suppose wff $\alpha$ and $\beta$ are tautology iff $\alpha \in \Delta$, $beta \in \Delta$.
			
			Then by the proposition we proved above $(\neg \alpha)$ is tautology iff any $\Sigma$ satisfies $(\neg \alpha)$ iff $(\neg \alpha) \in \Delta$. $(\alpha \circ \beta)$ is tautology iff any $\Sigma$ satisfies $(\alpha \circ \beta)$ iff $(\alpha \circ \beta) \in \Delta$ for any binary connective $\circ$.
			
			Thus, by the Principle of Induction, we conclude that for any wff $\varphi$, $\varphi$ is a tautology iff $\varphi \in \Delta$.
			
			$\blacksquare$
	\end{enumerate}
\end{document}